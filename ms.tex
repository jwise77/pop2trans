\documentclass[useAMS,usenatbib]{mn2e}

\usepackage{graphicx}
\usepackage{epsfig}
\usepackage{natbib}
\usepackage[section] {placeins}
\bibliographystyle{mn2e}
\citestyle{mn2e}

%%%%%%%% Begin custom definitions %%%%%%%%%%%%%

\input macros.tex
\voffset=-1.4cm

%%%%%%%% End custom definitions %%%%%%%%%%

\begin{document}

\title[Transition from Pop III to Dwarf Galaxies]{The Transition from
  Massive Metal Free Stars to Dwarf Galaxies}

\author[J. H. Wise et al.]{John H. Wise$^1$\thanks{Hubble Fellow;
    e-mail: jwise@astro.princeton.edu}, Matthew J. Turk$^2$, Michael
  L. Norman$^2$, Tom Abel$^3$\\
  $^{1}$ Department of Astrophysical Sciences, Princeton University,
  Princeton, NJ 08544, USA\\
  $^{2}$ Center for Astrophysics and Space Sciences,
  University of California at San Diego, La Jolla, CA 92093, USA\\
  $^{3}$ Kavli Institute for Particle Astrophysics and Cosmology,
  Stanford University, Menlo Park, CA 94025, USA}

\pagerange{\pageref{firstpage}--\pageref{lastpage}} \pubyear{2011}

\maketitle
\label{firstpage}

\begin{abstract}

  Population III stars first form in dark matter haloes with masses
  around $10^6 \Ms$.  By definition, they are metal-free, and their
  protostellar collapse is driven by molecular hydrogen cooling in the
  gas-phase, leading to a massive characteristic mass $\sim 100~\Ms$
  and suppressed fragmentation.  Population II stars with lower
  characteristic masses form when the star-forming gas reaches a
  critical metallicity of $10^{-6} - 10^{-3.5}~Z_\odot$, depending on
  whether dust cooling is important.  We present adaptive mesh
  refinement radiation hydrodynamics simulations that follows the
  transition from Population III to II star formation.  We model
  stellar radiative feedback with adaptive ray tracing.  A top-heavy
  initial mass function for the Population III stars is considered,
  resulting in a plausible distribution of pair-instability supernovae
  and associated metal enrichment.  We find that the gas fraction
  recovers from 5 percent to nearly the cosmic fraction in haloes with
  merger histories rich in haloes above $10^7 \Ms$.  A single
  pair-instability supernova is sufficient to enrich the host halo to
  a metallicity floor of $10^{-3} Z_\odot$ and to transition to
  Population II star formation.  This provides a natural explanation
  for the observed floor on damped Lyman alpha (DLA) systems
  metallicities reported in the literature, which is of this order.
  We find that stellar metallicities do not necessarily trace stellar
  ages, as mergers of haloes with established stellar populations can
  create superpositions of $t-Z$ evolutionary tracks.  A bimodal
  metallicity distribution is created after a starburst occurs when
  the halo can cool efficiently through atomic line cooling.

\end{abstract}

\begin{keywords}
  cosmology --- methods: numerical --- hydrodynamics ---
  radiative transfer --- star formation
\end{keywords}

\section{Introduction}

The first generation of stars, so-called Population (Pop) III, in the
universe were metal-free, forming from a primordial mixture of mainly
hydrogen and helium.  Heavy elements and dust grains were not present
in these objects to provide an efficient cooling mechanism in the
collapse of the host molecular cloud, whose collapses primarily rely
on molecular hydrogen line cooling in the gas-phase.  Without any
other cooling channels, the temperatures cannot cool below $\sim$ 200
K, resulting in Jeans masses on the order of 100 \Ms.  Numerical
calculations in the past decade have shown that Pop III stars
have characteristic masses between 30 and 300
\Ms~\citep[e.g.][]{ABN02, Bromm02_P3, OShea07a}, which form in dark
matter (DM) haloes with masses $\sim 10^6 \Ms$
\citep[e.g.][]{MacLow86, Shapiro87, Tegmark97}.

These stars have large luminosities on the order of 10$^6 \lsun$,
surface temperatures of $\sim 10^5$ K, and lifetimes of $\sim3$ Myr
\citep{Bond84, Schaerer02}.  Radiation during their main sequence
creates an \hii region with $r \sim 1-3~\mathrm{kpc}$, ionizing the
surrounding intergalactic medium as the virial radii of the host
haloes are $r_{\rm vir} \sim 100 \mathrm{pc}$ \citep{Whalen04,
  Kitayama04, Alvarez06, Abel07}.  The over-pressurized \hii region
drives a $\sim$ 30 \kms~shock, which is 10 times greater than the
escape velocity of a $10^6 \Ms$ halo.  In turn, most of the gas is
expelled from the halo, leaving behind a warm ($3 \times 10^4$ K) and
diffuse (0.5 \cubecm) medium.  In haloes that are encompassed by these
\hii regions and not completely photo-evaporated \citep{Shapiro04,
  whalen08}, the additional free electrons catalyze both \hh~and HD
molecular line cooling \citep{OShea05, Johnson06, Yoshida07,
  McGreer08}, cooling the gas below $\sim$ 100 K and lowering the
Jeans mass to $\sim$ 10--30 \Ms.  These lower-mass, metal-free stars
are called Pop III.2, whereas their more massive counterparts
that form in unheated regions are called Pop III.1 stars
\citep{Norman08}.

A fraction of Pop III stars enrich the surrounding IGM when
they go supernova, which can happen in stars $\lsim 40~\Ms$ in Type II
supernovae (SNe) or in stars roughly between 140 \Ms~and 260 \Ms~in
pair-instability SNe \citep[PISNe;][]{2002ApJ...567..532H}. The host
halo and the neighboring haloes are then enriched with this ejecta.
The blastwave expands to a radius of a few kpc \citep{Bromm03_SN,
  Wise08_Gal, Greif10} with approximately half of the ejecta falling
back into the adjacent filaments and halos.  The blastwave may induce
star formation in nearby haloes through the compression of the gas
\citep{Shapiro87, Ferrara98, Mackey03}, and the timescales for metal
mixing into the dense gas are many dynamical times \citep{Cen08} for
shock velocities $\lsim 100~\kms$.

When star-forming gas reaches some `critical metallicity' $Z_{\rm
  cr}$, radiative cooling from metal fine-structure lines or
\hh~formation on dust grains becomes efficient, allowing cooling to $T
\lsim 100$ K \citep{Omukai05}.  The exact value of $Z_{\rm cr}$ has
been constrained to be between $10^{-6} \zsun$ and $10^{-3.5} \zsun$.
The lower limit applies if dust cooling is important in these
scenarios \citep{Omukai05, Schneider06_Frag, clark08}, and the upper
limit happens if metal line cooling in the gas-phase is the dominant
process \citep{Bromm01, 2009ApJ...691..441S}.  This metal-enriched gas
fragments and most likely forms stars with masses similar to
present-day stellar initial mass functions (IMF), marking the local
transition from Pop III to II star formation.  Intermediate to
these populations and possibly unique to high redshift, there exists a
mode of massive star formation that has a characteristic mass of $\sim
10 \Ms$ \citep{Larson98, Tumlinson07_IMF, 2009ApJ...691..441S}.  This
happens when cooling in metal-enriched gas is limited to the cosmic
microwave background (CMB) temperature $T_{\rm CMB} = 2.73 (1+z)$ K,
which can be significantly higher than the cores of typical molecular
clouds ($T \sim 10$ K).

The transition from Pop III to Pop II star formation is solely
dependent on the propagation of metals from the SNe remnants into
future sites of star formation.  Their flows are complex because of
the interactions between the SN blastwave, cosmological accretion and
halo mergers, and nearby stellar feedback.  Spurred by the notion of a
critical metallicity, this transition has been extensively studied
with (i) volume-averaged semi-analytic models \citep{Scannapieco03,
  Yoshida04, Furlanetto05_Reion}, (ii) models using hierarchical
merger trees \citep{Tumlinson06, Tumlinson10, Salvadori07, Komiya10},
(iii) post-processing of cosmological simulations with blastwave
models \citep{Karlsson08, Trenti09, Trenti10}, and (iv) direct
numerical simulations with stellar feedback \citep{Tornatore07,
  Ricotti08, Maio10_Pop32, Maio10_Enrich}.

Over the past decade, these works have refined the general picture of
inhomogenous metal enrichment and the transition to Pop II star
formation, and here we give a brief overview of its development.
Considering only Pop III star formation, \citeauthor{Yoshida04}
calculated that Pop III stars can raise the mean metallicity to $-4.5
\lsim$ [Z/H]\footnote{We use the conventional notation, [Z/H] $\equiv
  \log(Z/H) - \log(Z_\odot/H_\odot)$.} $\lsim -3.5$ by redshift 15 in
the upper limit where all metal-free stars have $M = 200 \Ms$ and
produce PISNe.  Considering both Pop III and II star formation,
\citeauthor{Scannapieco03} found that the transition between the two
modes is a gradual process where both modes are coeval, confirmed by
most of the later works.  Because the host galaxies have small masses
and are subject to negative radiative feedback through photo-heating
and photo-dissociation, the minimum halo mass gradually increases with
the radiation background intensity, which
\citeauthor{Furlanetto05_Reion} found to delay metal enrichment and
place it closer to the epoch of reionization.  \citeauthor{Trenti09}
noted that underdense regions of the universe are still pristine at
$z=6$ with Pop III stars still forming at these late epochs.  This
group later expanded on these results to find that Pop II SFR becomes
dominant at $z>25$ in the buildup of a MW-type halo.  Furthermore they
stress the importance of a photo-dissociating radiation background
that reduces enrichment by PISNe and increases the importance of
metal-rich galactic outflows, where only $10^{-4} - 10^{-2}$ of PISN
ejecta ends up in EMP stars with $M > 0.9 \Ms$.

The Pop III IMF should play an important role in abundance patterns in
extremely metal poor (EMP) stars in the Milky Way (MW) halo and the
physics of reionization.  Using these data as constraints,
\citeauthor{Tumlinson06} found that Pop III IMFs with log-normal
distributions with mean masses between 8 and 42 \Ms~best fit the data.
Furthermore he concludes that EMP stars with [Z/H] $<$ --3 have
between 1 and 10 metal-free SN progenitors and the Pop III SFR is less
than 1\% of the total SFR at $z=6$.  \citeauthor{Karlsson08} use
observational data of EMP stars with their model to constrain the mass
fraction of Pop III stars that die with a PISN is less than 40 per
cent.  They also conclude that stars enriched primarily by PISNe have
[Ca/H] $\gsim$ --2.6, which could have been missed by some EMP
surveys.

%These first stellar clusters may be connected to stars in the Milky
%Way halo and nearby dwarf spheroidal (dSph) galaxies, both with a
%metallicity floor of [Z/H] = --4 \citep{Beers05, Tafelmeyer10,
%Frebel10_Obs}.
%

Cosmological simulations are useful but computationally expensive to
detangle and study the transition from Pop III to II star formation,
especially the complexities ensuing from hierarchical structure
formation, stellar feedback, and inhomogeneous chemical enrichment.
\citeauthor{Tornatore07} employed smoothed particle hydrodynamics
(SPH) simulations to study this problem and found that Pop III star
formation occurs down to $z=2.5$ in underdense regions.  Their mass
resolution did not allow them to capture star formation in minihalos
($M_{\rm halo} \sim 10^{5-7} \Ms$) that host Pop III stars, which may
have resulted in an underestimate of metal enrichment.  They found a
`Pop III wave' of star formation, where Pop III star formation is
quenched in an inside-out fashion from biased regions to voids, is
however still valid.  \citeauthor{Ricotti08} used a Lagrangian
moving-mesh simulation with radiative, mechanical, and chemical
feedback that resolved minihaloes to find that (i) 1--10\% of the
cosmic volume is chemically enriched, (ii) the gas reservoir in halos
with $M < 10^8 \Ms$ are depleted, (iii) there is a large scatter in
the mass-to-light (M/L) ratios of dwarf galaxies, and (iv) their
luminosity function is relatively flat.  Most recently
\citeauthor{Maio10_Pop32} found that the ratios of Pop III to Pop II
SFRs are insensitive to the exact value of $Z_{\rm cr}$.  In contrast
with \citeauthor{Ricotti08}, only $10^{-8}$ of the simulation volume
is chemically enriched by $z=11$.  This difference of several orders
of magnitude may result from the lack of radiative feedback, which can
expel most of the gas from the host halo before the SN explosion.
Supporting this idea, \citet{Whalen08_SN} found that PISN blastwaves
in neutral halos radiate all of their energy, stalling the shock
within the halo and suppressing outflows.

Here, we present simulations that include both modes of star formation
and their radiative, mechanical, and chemical feedback, extending the
aforementioned previous work on the transition from Pop III to II star
formation and the birth of the first galaxies.  The methods used here
incorporate and link together recent results from metal-enriched and
metal-free star formation, the critical metallicity, and
pair-instability supernovae.  There are some uncertainities in these
input parameters, and we thus run several simulations while varying
these parameters.  This is the first time it has been possible to link
the formation and feedback of the first stars to protogalaxies in a
numerical simulation, resolving the important scales and including the
most important physical effects.

This paper is structured as follows.  Section \ref{sec:setup}
describes our simulation setup, numerical methods for star formation
and feedback, and different physical models considered in each
simulation.  Next we present our results on the stellar populations
and metal enrichement of the high-redshift dwarf galaxies in Section
\ref{sec:results}.  Then in Section \ref{sec:models}, we show the
changes in galactic properties and metal enrichments in each physical
model.  In Section \ref{sec:compare}, we compare our results with
previous work on the transition to Pop II.  We then discuss the
importance of Pop III star formation on high-redshift galaxy
formation and the implications on the interpretation of damped
Lyman-$\alpha$ absorbers (DLAs) and galactic archaeology in Section
\ref{sec:discuss}.  Finally in Section \ref{sec:summary}, we summarize
our results.

\section{Method}
\label{sec:setup}

\begin{figure*}
  \epsscale{2}
  \plotone{bmosaic.eps}
  \epsscale{1}
  \caption{\label{fig:evo-mosaic} text.}
\end{figure*}

\subsection{Simulation setup}

We use the adaptive mesh refinement (AMR) code
\textsc{enzo~v2.0}\footnote{\texttt{enzo.googlecode.com, changeset
    b86d8ba026d6}} \citep{OShea2004}, which has been modified to use a
HLLC Riemann solver \citep{Toro94_HLLC} for additional stability in
strong shocks and rarefaction waves.  To resolve minihaloes with at
least 100 dark matter (DM) particles and follow the formation of the
first generation of dwarf galaxies, we use a simulation box of 1 Mpc
that has a resolution of $256^3$.  This gives us a DM mass resolution
of 1840 \Ms.  We refine the grid on baryon overdensities of $3 \times
2^{-0.2l}$, where $l$ is the AMR level, resulting in a
super-Lagrangian behavior.  We also refine on a DM overdensity of $3
\times 2^l$ and always resolve the local Jeans length by at least 4
cells, avoiding artificial fragmentation during gaseous collapses
\citep{Truelove97}.  This simulation has $1.4 \times 10^8$
computational elements and a maximal spatial resolution of 0.1 pc.  We
initialize the simulation with \textsc{grafic} \citep{Bertschinger01}
at $z = 130$ and use the cosmological parameters from the 7-year WMAP
$\Lambda$CDM+SZ+LENS best fit \citep{WMAP7}: $\Omega_M = 0.266$,
$\Omega_\Lambda = 0.734$, $\Omega_b = 0.0449$, $h = 0.71$, $\sigma_8 =
0.81$, and $n = 0.963$ with the variables having their usual
definitions.  We stop the simulation at $z=7$.

\subsection{Star formation}

We distinguish Pop II and Pop III SF by the total metallicity of the
densest cell in the molecular cloud.  Pop II stars are formed if [Z/H]
$> -4$, and Pop III stars are formed otherwise.  We do not consider
Pop III.2 stars and intermediate mass stars from CMB-limited cooling.

Simulations have shown that the characteristic mass of Pop III stars
$M_{\rm char} \sim 100~\Ms$.  They form in molecular clouds that
coexist with the dark matter halo center with limited fragmentation
occurring during their collapse; however \citet{2009Sci...325..601T}
and \citet{Stacy10_Binary} have recently shown that Pop III binaries
may form in a fraction of such haloes.

% Furthermore, detailed one-dimensional calculations have shown that
% their IMF should follow a Kroupa-like IMF that has a Salpeter slope
% at the high-mass end and an exponential cut-off below $M_{\rm char}$
% \cite{refs}.

For Pop III stars, we use the same SF model as \citet{Wise08_Gal} where
each star particle represents a single star, forming at an overdensity
of $5 \times 10^5$.  Instead of using a fixed stellar mass, we randomly
sampled from an IMF with a functional form of
%
\begin{equation}
\label{eqn:imf}
f(M)dM = M^{-1.3} \exp\left[-\left(\frac{M_{\rm
        char}}{M}\right)^{1.6}\right] dM
\end{equation}
to determine the stellar mass.  Above $M_{\rm char}$, it behaves as a
Salpeter IMF but is exponentially cutoff below that mass
\citep{Chabrier03, Clark09}.

For reproducibility, we record the number of times the random number
generator \citep[Mersenne twister;][]{MTwister} has been called for use
when restarting the simulations.  Although the Pop III star formation
history (SFH) in the entire simulation is identical for a given $N$
stars, the Pop III SFH for a given halo can be different between the
two simulations.  This happens because the order in which haloes form
Pop III stars can be altered from enhanced star formation and feedback
from nearby Pop II star clusters.

We treat Pop II SF with the same prescription as \citet{Wise09}, which
is a modified version of the \citet{Cen92} method but accounts for the
fact that the molecular clouds are resolved.  The critical overdensity
is the same as the Pop III SF model.  In each star-forming region,
seven percent of the cold gas ($T < 10^3$ K) is removed from the grid
and deposited into the star particle that lives for 20 Myr, the
maximum lifetime of an OB star.  These stars generate the majority of
the ionizing radiation and SNe feedback in stellar clusters, thus we
ignore lower mass stars.

\subsection{Stellar feedback}

We use mass-dependent luminosities and lifetimes of the Pop III stars
from \citet{Schaerer02}.  The radiation field is evolved with adaptive
ray tracing \citep{Abel02_RT, Wise10} and is coupled self-consistently
to the hydrodynamics.  We model the \hh~dissociating radiation with an
optically-thin, inverse square profile, centered on all stars.  These
stars die as pair-instability SNe (PISNe) if they are in the mass
range 140--260 \Ms~\citep{Heger03}.  We use the explosion energy from
\citet{Heger02}, where we fit the following function to their models,
$E_{\rm PISN} = 10^{51} \times [5.0 + 1.304 (M_{\rm He} - 64)]$, where
$M_{\rm He} = (13/24) \times (M_\star - 20) \Ms$ is the helium core
(and equivalently the ejecta) mass and $M_\star$ is the stellar mass.

The Pop II stars emit 6000 hydrogen ionizing photons per baryon over
their lifetime, and we do not consider singly- and doubly-ionizing
helium photons.  We note that low-metallicity stars generate up to a
factor of four more ionizing photons than a solar metallicity
population \citep{Schaerer03} and might be underestimating the
radiative feedback.  Nonetheless this study provides an excellent
first insight in the transition to Pop II SF, as the metal enrichment
is the key ingredient here.  For SN feedback, these stars generate
$6.8 \times 10^{48}$ erg s$^{-1}$ $\Ms^{-1}$ after living for 4 Myr,
which is injected into spheres with a radius of 10 pc.  If the
resolution of the grid is less than 10/3 pc, we deposit the energy
into a $3^3$ cube surrounding the star particle.

\subsection{Different physics configurations}

We perform five different simulations with the same initial
conditions.  The fiducial model considers (1) primordial cooling that
is solved with a non-equilbrium chemistry solver with 9 species of
hydrogen, helium, and molecular hydrogen, (2) a Pop III IMF
characteristic mass $M_{\rm char}$ = 100 \Ms, and (3) Pop II and Pop
III star formation.  We vary this model with the following:

\begin{enumerate}
\item Radiative cooling from metal species.  We calculate a cooling
  curve with CLOUDY \citep{CLOUDY} and use the method of
  \citep{2008MNRAS.385.1443S}.  Solar abundances were used in the
  cooling curve.  It is tabulated based on density, temperature,
  electron fraction, and metallicity.
\item Pop III IMF characteristic mass $M_{\rm char}$ = 40 \Ms.
\item Pop II star formation only.
\item Pop II star formation only and no molecular hydrogen cooling.
\end{enumerate}

\section{Results}
\label{sec:results}

\begin{figure*}
  \plottwo{z7-mosaic0.eps}{z7-mosaic1.eps}
  \caption{\label{fig:z7-mosaic} Redshift 7.}
\end{figure*}

\begin{figure*}
  \plottwo{z8-mosaic0.eps}{z8-mosaic1.eps}
  \caption{\label{fig:z8-mosaic} Redshift 8.}
\end{figure*}

\begin{figure*}
  \plottwo{z10-mosaic0.eps}{z10-mosaic1.eps}
  \caption{\label{fig:z10-mosaic} Redshift 10.}
\end{figure*}

\subsection{Halo mass evolution}

\begin{figure}
  \plottwo{history0.eps}{history5.eps}
  \caption{\label{fig:massevo} Mass evolution.}
\end{figure}

\subsection{Metal enrichment}

\begin{figure}
  \plotone{Zhistory.eps}
  \caption{\label{fig:Zevo} Metal evolution.}
\end{figure}

\begin{figure}
  \plottwo{z2.eps}{z3.eps}
  \caption{\label{fig:zhalo} Halo metallicities.}
\end{figure}

\begin{figure}
  \plotone{ztot.eps}
  \caption{\label{fig:ztot} Halo metallicities.}
\end{figure}

\subsection{Star formation}

\begin{figure}
  \plotone{mlratios.eps}
  \caption{\label{fig:ML} Mass to light ratios.}
\end{figure}

\begin{figure*}
  \epsscale{2.0}
  \plotone{sfh1.eps}
  \epsscale{1.0}
  \caption{\label{fig:SFR} Star formation history.}
\end{figure*}

\subsection{Negative feedback from gas loss}

\section{Differences between Physical Models}
\label{sec:models}

\section{Comparison to Previous Work}
\label{sec:compare}

\subsection{Semi-analytical}

\subsection{Numerical}

\section{Discussion}
\label{sec:discuss}

\li DLAs

\li Bimodality in [Z/H] distribution in globular clusters and dSphs

\li Importance of critical metallicity in satellite haloes, not in host
haloes

\li [$\alpha$/Fe] estimate

\li Importance of Pop III SF

\section{Summary}
\label{sec:summary}

\section*{Acknowledgements}

J.~H.~W. thanks Renyue Cen, Amina Helmi, Marco Spaans, and Eline
Tolstoy for enlightening discussions.  J.~H.~W. is supported by NASA
through Hubble Fellowship grant \#120-6370 awarded by the Space
Telescope Science Institute, which is operated by the Association of
Universities for Research in Astronomy, Inc., for NASA, under contract
NAS 5-26555.  M.~J.~T. and M.~L.~N. acknowledge partial support from
NASA ATFP grant NNX08AH26G.  Computational resources were provided by
NASA/NCCS award SMD-09-1439.  The majority of the analysis and plots
were done with \texttt{yt} \citep{yt_full_paper}.

\bibliography{ms}
\bsp
\label{lastpage}

\end{document}
