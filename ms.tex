\documentclass[apjl]{emulateapj}
%\documentclass[12pt,preprint]{aastex}

\usepackage{graphicx}
\usepackage{epsfig}
\usepackage{natbib}
%\usepackage[section] {placeins}
\bibliographystyle{apj}
\citestyle{apj}

%%%%%%%% Begin custom definitions %%%%%%%%%%%%%

\input macros.tex

%%%%%%%% End custom definitions %%%%%%%%%%

\begin{document}

\shorttitle{THE BIRTH OF A GALAXY}
\shortauthors{WISE ET AL.}

\title{The Birth of a Galaxy: Primoridal Metal Enrichment and
  Population II Stellar Populations}

\author{John H. Wise\altaffilmark{1,2}, 
  Matthew J. Turk\altaffilmark{3},
  Michael L. Norman\altaffilmark{3},
  Tom Abel\altaffilmark{4}}

\altaffiltext{1}{Department of Astrophysical Sciences, Princeton
  University, Peyton Hall, Ivy Lane, Princeton, NJ 08544}
\altaffiltext{2}{Hubble Fellow}
\altaffiltext{3}{Center for Astrophysics and Space Sciences,
  University of California at San Diego, La Jolla, CA 92093}
\altaffiltext{4}{Kavli Institute for Particle Astrophysics and
  Cosmology, Stanford University, Menlo Park, CA 94025}
\email{jwise@astro.princeton.edu}

\begin{abstract}

  Population III stars first form in dark matter halos with masses
  around $10^6 \Ms$.  By definition, they are metal-free, and their
  protostellar collapse is driven by molecular hydrogen cooling in the
  gas-phase, leading to a massive characteristic mass $\sim 100~\Ms$
  and suppressed fragmentation.  Population II stars with lower
  characteristic masses form when the star-forming gas reaches a
  critical metallicity of $10^{-6} - 10^{-3.5}~Z_\odot$, depending on
  whether dust cooling is important.  We present adaptive mesh
  refinement radiation hydrodynamics simulations that follows the
  transition from Population III to II star formation.  We model
  stellar radiative feedback with adaptive ray tracing.  A top-heavy
  initial mass function for the Population III stars is considered,
  resulting in a plausible distribution of pair-instability supernovae
  and associated metal enrichment.  We find that the gas fraction
  recovers from 5 percent to nearly the cosmic fraction in halos with
  merger histories rich in halos above $10^7 \Ms$.  A single
  pair-instability supernova is sufficient to enrich the host halo to
  a metallicity floor of $10^{-3} Z_\odot$ and to transition to
  Population II star formation.  This provides a natural explanation
  for the observed floor on damped Lyman alpha (DLA) systems
  metallicities reported in the literature, which is of this order.
  Neighboring halos are enriched to metallicities 10--100 times lower
  as the enriching blastwaves mainly bypass them.  Stellar
  metallicities trace gas metallicities within 0.2 dex.  We find that
  stellar metallicities do not necessarily trace stellar ages, as
  mergers of halos with established stellar populations can create
  superpositions of $t-Z$ evolutionary tracks.  A bimodal metallicity
  distribution is created after a starburst occurs when the halo can
  cool efficiently through atomic line cooling.

\end{abstract}

\keywords{cosmology --- methods: numerical --- hydrodynamics ---
  radiative transfer --- star formation}

\section{Motivation}

The first (Pop III) stars are metal-free and have a large
characteristic mass and suppressed fragmentation in its protostellar
collapse \citep{ABN02, Bromm02_P3, Yoshida03, OShea07a}.  A fraction of
these stars enrich the surrounding intergalactic medium (IGM) when
they go supernova, which can happen in stars $\lsim 40~\Ms$ in Type II
supernovae (SNe) or in stars roughly between 140 \Ms~and 260 \Ms~in
pair-instability SNe \citep[PISNe;][]{2002ApJ...567..532H}.  The host
halo and the neighboring halos are then enriched with this ejecta.
There exists a critical metallicity that is $\sim 10^{-6} Z_\odot$ if
dust cooling is efficient \citep{Omukai05, Schneider06_Frag, clark08}
and $\sim 10^{-3.5} Z_\odot$ otherwise \citep{Bromm01,
  2009ApJ...691..441S}, where the gas can cool rapidly, lowering its
Jeans mass.  An intermediate characteristic mass of $\sim10~\Ms$ can
be occur if the gas cooling is suppressed to the cosmic microwave
background (CMB) temperature \citep{Larson98, Tumlinson07_IMF,
  2009ApJ...691..441S}.  The resulting Population II star cluster will
thus have a lower characteristic stellar mass than its metal-free
progenitors.  These first stellar clusters may be connected to stars
in the Milky Way halo and nearby dwarf spheroidal (dSph) galaxies,
both with a metallicity floor of [Z/H]\footnote[\dag]{We use the
  conventional notation, [Z/H] $\equiv$ log(Z/H) -
  log(Z$_\odot$/H$_\odot$)} = --4 \citep{Beers05, Tafelmeyer10,
  Frebel10_Obs}.

The transition from Pop III to Pop II star formation (SF) is solely
dependent on the propagation of metals from the SNe remnants into
future sites of SF.  Their flows are complex because of the
interactions between the SN blastwave, cosmological accretion and halo
mergers, and nearby stellar feedback.  In minihalos ($M \sim
10^6~\Ms$), radiation from a massive Pop III star can drive a 30
\kms~shock, which is 10 times greater than the escape velocity of the
halo, and leaves behind a warm ($3 \times 10^4$ K) and diffuse (0.5
\cubecm) medium \citep{Kitayama04, Whalen04, Abel07}.  This aids in the
expansion of the blastwave because it delays the transition to the
Sedov-Taylor and snowplow phases.  In PISNe, approximately half of the
metals stay in the IGM with a metal bubble size of a few kpc
\citep{Wise08_Gal, Greif10}.  The blastwave may induce SF in nearby
halos through the compression of the gas \citep{Ferrara98}, and
timescales for metal mixing into the dense gas are many dynamical
times \citep{Cen08} for shock velocities $\lsim100~\kms$.

Numerical simulations are useful to detangle and study these
complexities and the transition from Pop III to II stars.  In this
Letter, we present the simulations that include both types of SF, and
their radiative and mechanical feedback.  The methods used here
incorporate and link together recent results from metal-enriched and
metal-free star formation, the critical metallicity, and
pair-instability supernovae.  This is the first time it has been
possible to link the formation and feedback of the first stars to
protogalaxies, resolving the important scales and including the most
important physical effects.  We describe our simulation setup and our
adopted SF models in \S \ref{sec:setup} and present our results in \S
\ref{sec:results}.  We discuss the implications of our findings and
conclude in the last section.

\section{Method}
\label{sec:setup}

In this section, we first describe our simulation setup.  We then
detail the SF models used here and how we model radiative and SNe
feedback.

\subsection{Simulation setup}

We use the adaptive mesh refinement (AMR) code
enzo~v2.0\footnote[\ddag]{\texttt{enzo.googlecode.com, unstable
    changeset b86d8ba026d6}} \citep{BryanNorman1997, OShea2004}, which
has been modified to use a HLLC Riemann solver \citep{Toro94_HLLC} for
additional stability in strong shocks.  To resolve minihalos with at
least 100 dark matter (DM) particles and follow the formation of the
first generation of dwarf galaxies, we use a simulation box of 1 Mpc
that has a resolution of $256^3$.  This gives us a DM mass resolution
of 1840 \Ms.  We refine the grid on baryon overdensities of $3 \times
2^{-0.2l}$, where $l$ is the AMR level, resulting in a
super-Lagrangian behavior.  We also refine on a DM overdensity of $3
\times 2^l$ and always resolve the local Jeans length by at least 4
cells, avoiding artificial fragmentation during gaseous collapses
\citep{Truelove97}.  We initialize the simulation with \textsl{grafic}
\citep{Bertschinger01} at $z = 130$ and use the cosmological parameters
from the 7-year WMAP data \citep{WMAP7}: $\Omega_M = 0.266$,
$\Omega_\Lambda = 0.734$, $\Omega_b = 0.0449$, $h = 0.71$, $\sigma_8 =
0.81$, and $n = 0.963$ with the variables having their usual
definitions.  We stop the simulation at $z=7$.

We use a non-equilibrium chemistry solver with 9 species of hydrogen,
helium, and molecular hydrogen \citep{Abel97}.  We spatially
distinguish metal enrichment from Pop II and Pop III stars.  We will
follow-up this study with one that considers radiative cooling from
metal-enriched gas, using the method of \citet{2008MNRAS.385.1443S}.

% We peform two simulations with the same initial conditions.  They
% differ only in that one considers primordial cooling, and the other
% considers additional cooling in metal-enriched gas.  For primordial
% cooling, we use a non-equilbrium chemistry solver with 9 species of
% hydrogen, helium, and molecular hydrogen.  For metal cooling, we use a
% cooling curve that is calculated from CLOUDY \citep{CLOUDY} and using
% the method of \citep{2008MNRAS.385.1443S}.  It is tabulated based on
% density, temperature, electron fraction, and metallicity.

\subsection{Star formation}

\begin{figure}
\epsscale{1.15}
\plotone{Lstar.eps}
\epsscale{1}
\caption{\label{fig:massfn} \textit{Top}: Halo mass function of the
  simulation compared with the analytical fit (solid) of
  \citet{Warren06}.  \textit{Bottom}: Total Pop II luminosities of
  halos with constant mass-to-light ratios overplotted.}
\end{figure}

We distinguish Pop II and Pop III SF by the total metallicity of the
densest cell in the molecular cloud.  Pop II stars are formed if [Z/H]
$> -4$, and Pop III stars are formed otherwise.  We do not consider
Pop III.2 stars and intermediate mass stars from CMB-limited cooling.

Simulations have shown that the characteristic mass of Pop III stars
$M_{\rm char} \sim 100~\Ms$.  They form in molecular clouds that
coexist with the dark matter halo center with limited fragmentation
occurring during their collapse; however \citet{2009Sci...325..601T}
and \citet{Stacy10_Binary} have recently shown that Pop III binaries
may form in a fraction of such halos.

% Furthermore, detailed one-dimensional calculations have shown that
% their IMF should follow a Kroupa-like IMF that has a Salpeter slope
% at the high-mass end and an exponential cut-off below $M_{\rm char}$
% \cite{refs}.

For Pop III stars, we use the same SF model as \citet{Wise08_Gal} where
each star particle represents a single star, forming at an overdensity
of $5 \times 10^5$.  Instead of using a fixed stellar mass, we randomly
sampled from an IMF with a functional form of
%
\begin{equation}
\label{eqn:imf}
f(M)dM = M^{-1.3} \exp\left[-\left(\frac{M_{\rm char}}{M}\right)^{1.6}\right]
\end{equation}
to determine the stellar mass.  Above $M_{\rm char}$, it behaves as a
Salpeter IMF but is exponentially cutoff below that mass
\citep{Chabrier03, Clark09}.  For reproducibility, we record the number
of times the random number generator (Mersenne twister;
\citet{MTwister}) has been called for use when restarting the
simulations.

%%% FOR THE COMPARISON BETWEEN RUNS WITH AND WITHOUT METAL COOLING %%%
%
% Although the Pop III star formation history (SFH) is identical for a
% given $N$ stars, the Pop III SFH for a given halo can be different
% between the two simulations.  This happens because the order in which
% halos form Pop III stars can be altered from enhanced star formation
% and feedback from nearby Pop II star clusters, as we will demonstrate
% in this Letter.

We treat Pop II SF with the same prescription as \citet{Wise09}, which
is a modified version of the \citet{Cen92} method but accounts for the
fact that the molecular clouds are resolved.  The critical overdensity
is the same as the Pop III SF model.  In each star-forming region,
seven percent of the cold gas ($T < 10^3$ K) is removed from the grid
and deposited into the star particle that lives for 20 Myr, the
maximum lifetime of an OB star.  These stars generate the majority of
the ionizing radiation and SNe feedback in stellar clusters, thus we
ignore lower mass stars.

\subsection{Stellar feedback}

\begin{figure*}
\epsscale{1.15}
\plotone{f1.eps}
\epsscale{1}
\caption{\label{fig:projections} Density-weighted projections of gas
  density (top), temperature (middle), and metallicity (bottom) at
  $z=7$.  The left column shows the entire simulation volume, where
  the center and right columns focus on the intense and quiet halos,
  which are marked by left and right arrows in the upper-left panel.
  The metallicity projections are a composite picture of metals
  originating from Pop III (red) and Pop II (blue) stars with magenta
  indicating a mixture of the two.}
\end{figure*}

We use mass-dependent luminosities and lifetimes of the Pop III stars
from \citet{Schaerer02}.  The radiation field is evolved with adaptive
ray tracing \citep{Abel02_RT, Wise10} and is coupled self-consistently
to the hydrodynamics.  We model the \hh~dissociating radiation with an
optically-thin, inverse square profile, centered on all stars.  These
stars die as pair-instability SNe (PISNe) if they are in the mass
range 140--260 \Ms~\citep{Heger03}.  We use the explosion energy from
\citet{Heger02}, where we fit the following function to their models,
$E_{\rm PISN} = 10^{51} \times [5.0 + 1.304 (M_{\rm He} - 64)]$, where
$M_{\rm He} = (13/24) \times (M_\star - 20) \Ms$ is the helium core
(and equivalently the ejecta) mass and $M_\star$ is the stellar mass.

The Pop II stars emit 6000 hydrogen ionizing photons per baryon over
their lifetime, and we do not consider singly- and doubly-ionizing
helium photons.  We note that low-metallicity stars generate up to a
factor of four more ionizing photons than a solar metallicity
population \citep{Schaerer03} and might be underestimating the
radiative feedback.  Nonetheless this study provides an excellent
first insight in the transition to Pop II SF, as the metal enrichment
is the key ingredient here.  For SN feedback, these stars generate
$6.8 \times 10^{48}$ erg s$^{-1}$ $\Ms^{-1}$ after living for 4 Myr,
which is injected into spheres with a radius of 10 pc.  If the
resolution of the grid is less than 10/3 pc, we deposit the energy
into a $3^3$ cube surrounding the star particle.

\begin{figure*}
\epsscale{1.15}
\plotone{f2.eps}
\epsscale{1}
\caption{\label{fig:evo} (a) Evolution of the total halo mass (top),
  stellar mass (middle), and gas fraction (bottom) of the quiet
  (dashed) and intense (solid) halos.  (b) Mass-weighted stellar
  metallicities and gas metallicities enriched by Pop II and Pop III
  SNe of the intense (top) and quiet (bottom) halos.}
\end{figure*}

\section{Results}
\label{sec:results}

Here we present the gaseous and stellar evolution of two selected
halos in the simulation: one that has an early mass buildup but no
major mergers after $z=12$, and one that experiences a series of major
mergers between $z=10$ and $z=7$.  We name the halos ``quiet'' and
``intense'', respectively.  The entire simulation contains 38 galaxies
with 3640 Pop II stellar clusters and captures the formation of 333
Pop III stars.  The halo mass function and galaxy luminosities are
plotted in Figure \ref{fig:massfn}.  This simulation has $1.4 \times
10^8$ computational elements and a maximal spatial resolution of 0.1
pc.

% We start with the mass accretion history and enrichment of the
% halos.  We then compare the nature of the stellar populations in
% these halos.

We illustrate the state of the simulation at $z=7$ in Figure
\ref{fig:projections} with density weighted projections of gas
density, temperature, and metallicity, showing the entire box and
focusing on the two halos of interest.  Radiative and mechanical
feedback create a multi-phase medium inside these halos, which are
embedded in a warm and ionized IGM.

\begin{figure*}
\epsscale{1.1}
\plotone{f3.eps}
\epsscale{1}
\caption{\label{fig:pops} The scatter plots show the SF history of the
  quiet (left) and intense (right) halos as a function of metallicity
  at $z=7$.  Each circle represents a star cluster, whose area is
  proportional to its mass.  The open circles in the upper right
  represent sizes of $10^3$, $10^4$, and $10^5$ \Ms~star clusters.
  The dashed lines in the right panel guide the eye to two stellar
  populations that were formed in two satellite halos, merging at
  $z=7.5$.  The upper histogram shows the SF rate.  The right
  histogram depicts the stellar metallicity distribution.}
\end{figure*}

\subsection{Mass evolution}
\label{sec:halo}

%\li Describe the evolution of the baryon fraction and metallicity from
%PISN and Type II SNe metals in the two halos.

Figure \ref{fig:evo}a shows the total, metal-enriched stellar, and gas
mass history of the most massive progenitors of both halos.  The quiet
halo undergoes a series of major mergers at $z > 12$, growing by a
factor of 30 to $2.5 \times 10^7 \Ms$ within 150 Myr.  Afterwards it
only grows by a factor of 3 by $z=7$ mainly through smooth accretion
from the filaments and IGM.  It is the most massive halo in the
simulation between redshifts 13 and 10.  At the same time, the intense
halo has a mass $M = 3 \times 10^6 \Ms$, but it is contained in a
biased region on a comoving scale of 50 kpc with $\sim25$ halos with
$M \sim 10^6 \Ms$.  After $z=10$, these halos hierarchically merge to
form a $10^9 \Ms$ halo at $z=7$ with two major mergers at redshifts 10
and 7.9, seen in the rapid increases in total mass in Figure
\ref{fig:evo}a.  The merger history of the two halos are not atypical
as dark matter halos can experience both quiescent and vigorous mass
accretion rates.

Both halos start forming Pop II stars when $M = 10^7 \Ms$.  This is
consistent with the filtering mass $M_f$ of high-redshift halos when
it accretes mainly from a pre-heated IGM \citep{gnedin98, gnedin00,
  Wise08_Gal}.  Afterwards these halos can cool efficiently through
\hh~cooling, sustaining constant and sometimes bursting SF.  The
latter characteristics are equated with the definition of a galaxy.
The quiet halo forms $10^5 \Ms$ of stars by $z=9$.  This initial
starburst photo-evaporates the majority of its molecular clouds, in
addition to heating and ionizing the surrounding IGM out to a radius
of 10--15 kpc at $z=9$.  These respectively reduce the in-situ and
external cold gas supply that could feed future SF.

The gas fractions of both halos decrease from 0.15 to 0.08 by outflows
driven by ionization fronts and blastwaves in their initial
starbursts.  The quiet halo does not have a major merger with any halo
with $M > M_f$, leading to a small final gas fraction.  These low-mass
halos are photo-evaporated, hosting diffuse warm gas reservoirs
instead of cold dense cores.  After $z=10$, the halo mainly accretes
warm diffuse gas from the filaments and IGM.  In contrast, the intense
halo grows from major mergers of halos with $M > M_f$.  The progenitor
halos involved in the major mergers are able to host molecular clouds
and have higher gas fractions.  Between $z=10$ and $z=8$, the gas
fraction increases from 0.07 to 0.12 until it jumps to 0.14 when a
gas-rich major merger occurs.  The stellar mass accordingly increases
with the ample supply of gas during this period.  

\subsection{Metallicity evolution}

The evolution of the stellar and gas metallicity of both halos are
illustrated in Figure \ref{fig:evo}b.  PISNe from Pop III stars enrich
the nearby IGM out to a radius of 10 kpc and provides a metallicity
floor of $[Z_3/H] \sim -3$.  The metallicity from Pop II SNe initially
enrich the ISM of both halos to an average $[\bar{Z}_2/H]$ between
--3.5 and --3.  

In the quiet halo, an equilibrium of $[\bar{Z}_2/H] \sim -2.5$ is
established between metal-rich outflows and metal-poor inflows.
Galactic outflows are directed in the polar directions of the gas
disk, keeping the adjacent filaments metal-poor.  These features and a
well-mixed ISM \citep[cf.][]{Wise08_Gal, Greif10} are apparent in the
metallicity projections in Figure \ref{fig:projections}.  The average
stellar metallicity is within 0.5 dex of $[\bar{Z}_2/H]$.

In the intense halo, the first few Pop II star clusters have $[Z/H]$
between --1 and --2 and dominate the average stellar metallicity at $z
> 8$.  Afterwards the metallicity increases by a factor of 30 to
$[\bar{Z}_2/H] = -1.5$ through self-enrichment from a starburst.
Because this halo is located in a large-scale overdensity, most of the
ejecta falls back into the halo after reaching distances up to 20
comoving kpc, keeping the halo gas metallicity high because the
inflows are relatively metal-rich themselves.  After the $z \sim 8$
starburst, the average stellar metallicity follows the average gas
metallicity within 0.1 dex.

\subsection{Star formation history}
\label{sec:pop}

The most massive progenitor of the quiet halo interestingly never
hosts a Pop III star.  Instead a nearby halo forms a Pop III star,
which (randomly) produces a PISNe at $z=16$.  The blastwave overruns
the most massive progenitor, and the dense core survives this event
and is enriched by this PISN, triggering the transition to Pop II SF.
Other progenitors host three Pop III stars, forming at $z = 15.4,
14.2, 13.8$, with the latter producing a PISN.  Metal enrichment from
these two PISNe and Pop II SF quench Pop III SF in this halo.  The
progenitors of the intense halo host a total of 56 Pop III stars with
21 producing PISNe.  The first forms in a $6 \times 10^5 \Ms$ halo at
$z=19$.  Pop III stars form on a regular interval in the halo's
progenitors until $z=9$ when most of these halos enter the metal-rich
bubble surrounding the intense halo.

%% \li Present results on the star formation history and the physical
%% reasons for the features seen in the SFH and metallicities.
%%
%% \li Present the stellar metallicity distribution at the final time

Figure \ref{fig:pops} shows the SF history (SFH), metallicity
distribution, and SF rates of both halos.  A nearby PISNe provides a
metallicity floor of [Z/H] = --2.8 in the quiet halo at which
metallicity the first Pop II stars form.  The stellar metallicity
evolution exhibits what is expected from an isolated system with the
stellar feedback steadily enriching the ISM, resulting in a
correlation between stellar age and metallicity.  After $z=10$ the
metallicities plateau at [Z/H] = --2.1 for reasons previously
discussed.  The SFR peaks at $z=10$ and decreases as the cold gas
reservoir is depleted.  Around $z=7.5$, a 25:1 minor merger occurs,
and the gas inside the satellite halo is compressed, triggering
metal-poor, [Z/H] = --3.2, SF during its nearest approach.  This halo
remains metal-poor because most of metal enrichment in the quiet halo
occurs in bi-polar flows perpendicular to the galaxy disk and
filament.  Stars with [Z/H] $<$ --3 compose 1.6 percent of the total
stellar mass.
       
In contrast with the quiet halo, the intense halo undergoes a few
mergers of halos with an established stellar population.  This creates
a superposition of age-metallicity tracks in the SFH, seen in the
complexity of Figure \ref{fig:pops}.  The first two Pop II stellar
clusters have an unexpectedly high metallicity [Z/H] $\sim$ --1, which
occurs when a PISN blastwave triggers SF in two neighboring halos.
Most of the early SF have [Z/H] = --2.5.  At $z=9$, the halo's virial
temperature reaches \tento{4} K.  This combined with a 10:1 merger
creates a starburst that quickly enriches the halo to [Z/H] = --1.5 by
$z=8$.  The halo continues to enrich itself afterwards.  The spikes in
the scatter plot correspond to SN triggered SF in nearby molecular
clouds that are enriched up to a factor of 10 with respect to the ISM.
However their mass fraction are small compared to the total stellar
mass.  The starburst at $z=9$ creates a bimodal metallicity
distribution with peaks at [Z/H] = --2.4 and --1.2 with the metal-rich
component mainly being created after the starburst.  Two systems with
sizable stellar components merge into the halo at $z \sim 8$, and
their stellar populations are still discernible in the metallicity-age
plot.  Stars with [Z/H] $<$ --3 compose 1.8 percent of the total
stellar mass.

\section{Discussion and Conclusions}

In this Letter, we focus on the birth of two galaxies prior to
reionization with a cosmological AMR radiation hydrodynamics
simulation.  Supernovae from Pop III stars provide the necessary heavy
elements for the transition to a Population II stellar population,
which we have directly simulated.  These two galaxies have a 10--15\%
probability in surviving as present-day ``fossil'' galaxies
\citep{Gnedin06}, otherwise they will be incorporated into galactic
stellar halos.  A $z=8.55$ galaxy was recently spectroscopically
confirmed that is contained in an ionized bubble with radius 0.1--0.5
Mpc with an uncharacterized population of galaxies contributing to the
local ionizing radiation field \citep{Lehnert10_z8.6}.  The galaxies
simulated here might represent this population.

We find that one PISN is sufficient to enrich the star-forming halo
and surrounding $\sim 5$ kpc to a metallicity of 10$^{-3} Z_\odot$,
given $M_{\rm char} = 100~\Ms$.  DLA systems have a metallicity floor
on the same order \citep{Wolfe05_Review}, and metal enrichment from Pop
III SNe provides a possible explanation.

If the first stars have a lower characteristic mass that favor
hypernovae \citep{Tumlinson07_IMF}, then this metallicity floor should
be lowered by a factor of $\sim 10$ because (1) the metal ejecta is
lowered by a factor of $\sim 50$ and (2) the mixing mass is
approximately decreased by a factor of $(E_{\rm hyp}/E_{\rm
  PISN})^{3/5}$ in the Sedov-Taylor solution, where $(E_{\rm
  hyp}/E_{\rm PISN}) \sim 0.1$ is the ratio of explosion energies of a
hypernova and PISN.  In the case where this metallicity floor is less
than the critical metallicity, then the next instance of SF will
further enrich the ISM, solidifying the transition to Pop II SF.  We
conclude that it only takes one, at most two, SNe from Pop III stars
in the halo progenitors to complete the transition to Population II
\citep{Frebel10}.  The question of whether the critical metallicity is
\tento{-6} or \tento{-3.5} $Z_\odot$ is most applicable to nearby
halos where the heavy elements mix slowly into dense cores as the
blastwave overtakes it.  Less than two percent of stars have [Z/H] $<
-3$ in both systems, consistent with observations of metal-poor stars
in the halo and dSphs \citep{Beers05, Battaglia10}.

% We also find that the merger history plays an important role in
% supplying gas into halos after its first epoch of SF.  Mergers of
% halos below the filtering mass are inefficient in providing gas
% whereas the opposite is true for merging halos above the filtering
% mass.  Halos do not necessarily need $\tvir \ge 10^4$ K to form
% significant stellar populations; however SFRs dramatically increase,
% and thus metal enrichment, when this threshold is reached.  Halos
% with mostly gas-poor mergers or a quiet merger history result in a
% monotonic increase in metallicity with time.  SFHs become more
% complex with multiple metallicity-age tracks if the halo experiences
% mergers with other halos that have an established stellar
% population.  Furthermore SF that is triggered by blastwaves
% interacting with molecular clouds can have metallicities up to a
% factor of 10 higher than the main starburst.  Starbursts at $\tvir =
% 10^4$ K can enrich the host halo enough to create a bimodal
% metallicity distribution, where the metal-poor component is created
% before the burst.  Note that mergers of stellar populations can also
% create a similar bimodal distribution, which are observed in dSphs
% \cite{Battaglia10}.

We have shown that it is possible to simulate the formation of a
high-redshift dwarf galaxy and its entire SFH with radiative and
mechanical feedback.  These results provide invaluable insight on the
first galaxies and the role of metal-free stars in the early universe.
There exists a wealth of information in this simulation, and we plan
to follow up this preliminary report with more detailed analysis of
the metal enrichment of the IGM, global SF rates, and observational
connections with high-redshift galaxies and Local Group dwarf galaxies
in the near future.

\acknowledgments

Support for this work was provided by NASA through Hubble Fellowship
grant \#120-6370 awarded by the Space Telescope Science Institute,
which is operated by the Association of Universities for Research in
Astronomy, Inc., for NASA, under contract NAS 5-26555.  Computational
resources were provided by NASA/NCCS award SMD-09-1439.
J.~H.~W. thanks Renyue Cen, Amina Helmi, Marco Spaans, and Eline
Tolstoy for enlightening discussions.  The majority of the analysis
and plots were done with \texttt{yt} \citep{yt_full_paper}.

%\clearpage
%\input biblio.tex
\bibliography{ms}

\end{document}
