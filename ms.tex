\documentclass[apjl]{emulateapj}
%\documentclass[12pt,preprint]{aastex}

\usepackage{graphicx}
\usepackage{epsfig}
\usepackage{natbib}
\usepackage[section] {placeins}
\bibliographystyle{apj}
\citestyle{apj}

%%%%%%%% Begin custom definitions %%%%%%%%%%%%%

\input macros.tex

%%%%%%%% End custom definitions %%%%%%%%%%

\begin{document}

\shorttitle{TRANSITION TO POPULATION II}
\shortauthors{WISE ET AL.}

\title{How Short Lived is the Population III Epoch?}

\author{John H. Wise\altaffilmark{1,2}, 
  Michael L. Norman\altaffilmark{3},
  Tom Abel\altaffilmark{4}}

\altaffiltext{1}{Department of Astrophysical Sciences, Princeton
  University, Peyton Hall, Ivy Lane, Princeton, NJ 08544}
\altaffiltext{2}{Hubble Fellow}
\altaffiltext{3}{Center for Astrophysics and Space Sciences,
  University of California at San Diego, La Jolla, CA 92093}
\altaffiltext{4}{Kavli Institute for Particle Astrophysics and
  Cosmology, Stanford University, Menlo Park, CA 94025}
\email{jwise@astro.princeton.edu}

\begin{abstract}

  Population III stars first form in dark matter halos with masses
  around $10^6 \Ms$.  By definition, they are metal-free, and their
  protostellar collapse is driven by molecular hydrogen cooling in the
  gas-phase, leading to a massive characteristic mass $\sim 100~\Ms$
  and suppressed fragmentation.  Population II stars with lower
  characteristic masses form when the star-forming gas reaches a
  critical metallicity of $10^{-6} - 10^{-3}~Z_\odot$, depending on
  whether dust cooling is important.  We present adaptive mesh
  refinement radiation hydrodynamics simulations that
  self-consistently follow the transition from Population III to II
  star formation.  We model stellar radiative feedback with adaptive
  ray tracing.  A top-heavy initial mass function for the Population
  III stars is considered, resulting in a plausible distribution of
  pair-instability supernovae and associated metal enrichment.
  \textbf{Add results.}

\end{abstract}

\keywords{cosmology --- methods: numerical --- hydrodynamics ---
  radiative transfer --- star formation}

\section{Motivation}

The first (Pop III) stars are metal-free and have a large
characteristic mass and suppressed fragmentation in its protostellar
collapse \citep{refs}.  A fraction of these stars enrich the
surrounding intergalactic medium (IGM) when they go supernova, which
can happen in stars $\lsim40~\Ms$ in Type II supernovae (SNe) or in
stars roughly between 140 \Ms~and 260 \Ms~in pair-instability SNe
\citep[PISNe;][]{Heger02}.  The host halo and the neighboring halos
are then enriched with this ejecta.  There exists a critical
metallicity that is $\sim 10^{-6} Z_\odot$ if dust cooling is
efficient \citep{refs} and $\sim 10^{-3.5} Z_\odot$ otherwise
\citep{refs}, where the gas can cool rapidly, lowering its Jeans mass.
An intermediate characteristic mass of $\sim10~\Ms$ can be occur if
the gas cooling is suppressed to the cosmic microwave background (CMB)
temperature \citep{ref}.  The resulting Population II star cluster
will thus have a lower characteristic stellar mass than its metal-free
progenitors \citep{refs}.  In the local universe, stars in the Milky
Way halo and dwarf spheroidal (dSph) galaxies may be connected to
these first stellar clusters.  In the halo of the Milky Way, the most
metal-poor stars have a metallicity floor of [Z/H] = ??? \citep{ref},
and in dwarf spheroidal (dSph) galaxies this floor is a factor of 10
higher \citep{refs}.

The transition from Pop III to Pop II star formation is solely
dependent on the propagation of metals from the SNe remnants into
future sites of star formation.  Their flows are complex because of
the interactions between the SN blastwave, cosmological accretion and
halo mergers, and nearby stellar feedback.  In minihalos ($M_{\rm h}
\sim 10^6~\Ms$), radiation from the massive Pop III drives a 30
\kms~shock, which is 10 times greater than the escape velocity of the
halo, and leaves behind a warm ($3 \times 10^4$ K) and diffuse (0.5
\cubecm) medium \citep{refs}.  The latter feature aids in the
expansion of the blastwave because it delays the transition to the
Sedov-Taylor and snowplow phases.  In PISNe, approximately half of the
metals stay in the IGM with an metal bubble size of a few kpc.  The
remaining metals fall back into the host halo.  The blastwave may
induce star formation in nearby halos through the compression of the
gas \citep{refs}; however timescales for metal mixing into the dense
gas are many dynamical times \citep{refs} for shock velocities $\lsim
100~\kms$.

Numerical simulations are useful to detangle and study these
complexities and the transition from Pop III to II stars.  In this
Letter, we present the simulations that include both types of star
formation, its radiative feedback, and its SNe feedback.  We utilize
an initial mass function (IMF) for individual Pop III stars, based on
the latest simulation results.  The methods used here incorporate and
link together recent results from metal-enriched and metal-free star
formation, the critical metallicity, pair-instability supernovae, and
fine-structure metal line cooling.  Furthermore we spatially
distinguish metal enrichment from Pop II and Pop III stars as their
abundance patterns differ significantly \citep{ref}.  This is the
first time that all of these processes have been interconnected in a
simulation that resolves minihalos to study the transition to Pop II
star formation self-consistently.  We describe our simulation setup
and our adopted star formation models in \S \ref{sec:setup} and
present our results in \S \ref{sec:results}.  We discuss the
implications of our findings and conclude in the last section.

\section{Simulation Setup}
\label{sec:setup}

\section{Results}
\label{sec:results}

\section{Discussion and Summary}

\acknowledgments

J.H.W. is supported by the Hubble Fellowship etc.  The majority of the
analysis and plots were done with \texttt{yt} \citep{yt}.

%\clearpage
%\input biblio.tex
\bibliography{ms}

\end{document}
