%\documentclass[apjl]{emulateapj}
\documentclass[12pt,preprint]{aastex}

\usepackage{graphicx}
\usepackage{epsfig}
\usepackage{natbib}
\usepackage[section] {placeins}
\bibliographystyle{apj}
\citestyle{apj}

%%%%%%%% Begin custom definitions %%%%%%%%%%%%%

\input macros.tex

%%%%%%%% End custom definitions %%%%%%%%%%

\begin{document}

\shorttitle{High Redshift Dwarf Galaxies}
\shortauthors{WISE ET AL.}

\title{Metal Enrichment and Stellar Populations of High Redshift Dwarf
  Galaxies}

\author{John H. Wise\altaffilmark{1,2}, 
  Matthew J. Turk\altaffilmark{3},
  Michael L. Norman\altaffilmark{3},
  Tom Abel\altaffilmark{4}}

\altaffiltext{1}{Department of Astrophysical Sciences, Princeton
  University, Peyton Hall, Ivy Lane, Princeton, NJ 08544}
\altaffiltext{2}{Hubble Fellow}
\altaffiltext{3}{Center for Astrophysics and Space Sciences,
  University of California at San Diego, La Jolla, CA 92093}
\altaffiltext{4}{Kavli Institute for Particle Astrophysics and
  Cosmology, Stanford University, Menlo Park, CA 94025}
\email{jwise@astro.princeton.edu}

\begin{abstract}

  Population III stars first form in dark matter halos with masses
  around $10^6 \Ms$.  By definition, they are metal-free, and their
  protostellar collapse is driven by molecular hydrogen cooling in the
  gas-phase, leading to a massive characteristic mass $\sim 100~\Ms$
  and suppressed fragmentation.  Population II stars with lower
  characteristic masses form when the star-forming gas reaches a
  critical metallicity of $10^{-6} - 10^{-3.5}~Z_\odot$, depending on
  whether dust cooling is important.  We present adaptive mesh
  refinement radiation hydrodynamics simulations that follows the
  transition from Population III to II star formation.  We model
  stellar radiative feedback with adaptive ray tracing.  A top-heavy
  initial mass function for the Population III stars is considered,
  resulting in a plausible distribution of pair-instability supernovae
  and associated metal enrichment.  We find that the gas fraction
  recovers from 5 percent to nearly the cosmic fraction in halos with
  merger histories rich in halos above $10^7 \Ms$.  A single
  pair-instability supernova is sufficient to enrich the host halo to
  a metallicity floor of $10^{-3} Z_\odot$ and to transition to
  Population II star formation.  This provides a natural explanation
  for the observed floor on damped Lyman alpha (DLA) systems
  metallicities reported in the literature, which is of this order.
  We find that stellar metallicities do not necessarily trace stellar
  ages, as mergers of halos with established stellar populations can
  create superpositions of $t-Z$ evolutionary tracks.  A bimodal
  metallicity distribution is created after a starburst occurs when
  the halo can cool efficiently through atomic line cooling.

\end{abstract}

\keywords{cosmology --- methods: numerical --- hydrodynamics ---
  radiative transfer --- star formation}

\section{Motivation}

The first (Pop III) stars are metal-free and have a large
characteristic mass and suppressed fragmentation in its protostellar
collapse \citep{ABN02, Bromm02_P3, OShea07a}.  A fraction of these
stars enrich the surrounding intergalactic medium (IGM) when they go
supernova, which can happen in stars $\lsim 40~\Ms$ in Type II
supernovae (SNe) or in stars roughly between 140 \Ms~and 260 \Ms~in
pair-instability SNe \citep[PISNe;][]{2002ApJ...567..532H}.  The host
halo and the neighboring halos are then enriched with this ejecta.
There exists a critical metallicity that is $\sim 10^{-6} Z_\odot$ if
dust cooling is efficient \citep{Omukai05, Schneider06_Frag, clark08}
and $\sim 10^{-3.5} Z_\odot$ otherwise \citep{Bromm01,
  2009ApJ...691..441S}, where the gas can cool rapidly, lowering its
Jeans mass.  An intermediate characteristic mass of $\sim10~\Ms$ can
be occur if the gas cooling is suppressed to the cosmic microwave
background (CMB) temperature \citep{Larson98, Tumlinson07_IMF,
  2009ApJ...691..441S}.  The resulting Population II star cluster will
thus have a lower characteristic stellar mass than its metal-free
progenitors.  These first stellar clusters may be connected to stars
in the Milky Way halo and nearby dwarf spheroidal (dSph) galaxies,
both with a metallicity floor of [Z/H] = --4 \citep{Beers05,
  Tafelmeyer10, Frebel10_Obs}.

The transition from Pop III to Pop II star formation (SF) is solely
dependent on the propagation of metals from the SNe remnants into
future sites of SF.  Their flows are complex because of the
interactions between the SN blastwave, cosmological accretion and halo
mergers, and nearby stellar feedback.  \textbf{Discuss prior work on
  transition with the papers given by the previous referee.}

In minihalos ($M \sim 10^6~\Ms$), radiation from a massive Pop III
star can drive a 30 \kms~shock, which is 10 times greater than the
escape velocity of the halo, and leaves behind a warm ($3 \times 10^4$
K) and diffuse (0.5 \cubecm) medium \citep{Kitayama04, Whalen04,
  Abel07}.  This aids in the expansion of the blastwave because it
delays the transition to the Sedov-Taylor and snowplow phases.  In
PISNe, approximately half of the metals stay in the IGM with a metal
bubble size of a few kpc \citep{Wise08_Gal, Greif10}.  The blastwave
may induce SF in nearby halos through the compression of the gas
\citep{Ferrara98}, and timescales for metal mixing into the dense gas
are many dynamical times \citep{Cen08} for shock velocities
$\lsim100~\kms$.

Numerical simulations are useful to detangle and study these
complexities and the transition from Pop III to II stars.  In this
Letter, we present a simulation that includes both types of SF, and
their radiative and mechanical feedback.  The methods used here
incorporate and link together recent results from metal-enriched and
metal-free star formation, the critical metallicity, and
pair-instability supernovae.  This is the first time it has been
possible to link the formation and feedback of the first stars to
protogalaxies, resolving the important scales and including the most
important physical effects.

\section{Method}
\label{sec:setup}

\subsection{Simulation setup}

We use the adaptive mesh refinement (AMR) code
enzo~v2.0\footnote{\texttt{enzo.googlecode.com, changeset
    b86d8ba026d6}} \citep{OShea2004}, which has been modified to use a
HLLC Riemann solver \citep{Toro94_HLLC} for additional stability in
strong shocks and rarefaction waves.  To resolve minihalos with at
least 100 dark matter (DM) particles and follow the formation of the
first generation of dwarf galaxies, we use a simulation box of 1 Mpc
that has a resolution of $256^3$.  This gives us a DM mass resolution
of 1840 \Ms.  We refine the grid on baryon overdensities of $3 \times
2^{-0.2l}$, where $l$ is the AMR level, resulting in a
super-Lagrangian behavior.  We also refine on a DM overdensity of $3
\times 2^l$ and always resolve the local Jeans length by at least 4
cells, avoiding artificial fragmentation during gaseous collapses
\citep{Truelove97}.  This simulation has $1.4 \times 10^8$
computational elements and a maximal spatial resolution of 0.1 pc.  We
initialize the simulation with \textsl{grafic} \citep{Bertschinger01}
at $z = 130$ and use the cosmological parameters from the 7-year WMAP
$\Lambda$CDM+SZ+LENS best fit \citep{WMAP7}: $\Omega_M = 0.266$,
$\Omega_\Lambda = 0.734$, $\Omega_b = 0.0449$, $h = 0.71$, $\sigma_8 =
0.81$, and $n = 0.963$ with the variables having their usual
definitions.  We stop the simulation at $z=7$.

\subsection{Star formation}

We distinguish Pop II and Pop III SF by the total metallicity of the
densest cell in the molecular cloud.  Pop II stars are formed if [Z/H]
$> -4$, and Pop III stars are formed otherwise.  We do not consider
Pop III.2 stars and intermediate mass stars from CMB-limited cooling.

Simulations have shown that the characteristic mass of Pop III stars
$M_{\rm char} \sim 100~\Ms$.  They form in molecular clouds that
coexist with the dark matter halo center with limited fragmentation
occurring during their collapse; however \citet{2009Sci...325..601T}
and \citet{Stacy10_Binary} have recently shown that Pop III binaries
may form in a fraction of such halos.

% Furthermore, detailed one-dimensional calculations have shown that
% their IMF should follow a Kroupa-like IMF that has a Salpeter slope
% at the high-mass end and an exponential cut-off below $M_{\rm char}$
% \cite{refs}.

For Pop III stars, we use the same SF model as \citet{Wise08_Gal} where
each star particle represents a single star, forming at an overdensity
of $5 \times 10^5$.  Instead of using a fixed stellar mass, we randomly
sampled from an IMF with a functional form of
%
\begin{equation}
\label{eqn:imf}
f(M)dM = M^{-1.3} \exp\left[-\left(\frac{M_{\rm
        char}}{M}\right)^{1.6}\right] dM
\end{equation}
to determine the stellar mass.  Above $M_{\rm char}$, it behaves as a
Salpeter IMF but is exponentially cutoff below that mass
\citep{Chabrier03, Clark09}.

For reproducibility, we record the number of times the random number
generator (Mersenne twister; \citet{MTwister}) has been called for use
when restarting the simulations.  Although the Pop III star formation
history (SFH) in the entire simulation is identical for a given $N$
stars, the Pop III SFH for a given halo can be different between the
two simulations.  This happens because the order in which halos form
Pop III stars can be altered from enhanced star formation and feedback
from nearby Pop II star clusters.

We treat Pop II SF with the same prescription as \citet{Wise09}, which
is a modified version of the \citet{Cen92} method but accounts for the
fact that the molecular clouds are resolved.  The critical overdensity
is the same as the Pop III SF model.  In each star-forming region,
seven percent of the cold gas ($T < 10^3$ K) is removed from the grid
and deposited into the star particle that lives for 20 Myr, the
maximum lifetime of an OB star.  These stars generate the majority of
the ionizing radiation and SNe feedback in stellar clusters, thus we
ignore lower mass stars.

\subsection{Stellar feedback}

We use mass-dependent luminosities and lifetimes of the Pop III stars
from \citet{Schaerer02}.  The radiation field is evolved with adaptive
ray tracing \citep{Abel02_RT, Wise10} and is coupled self-consistently
to the hydrodynamics.  We model the \hh~dissociating radiation with an
optically-thin, inverse square profile, centered on all stars.  These
stars die as pair-instability SNe (PISNe) if they are in the mass
range 140--260 \Ms~\citep{Heger03}.  We use the explosion energy from
\citet{Heger02}, where we fit the following function to their models,
$E_{\rm PISN} = 10^{51} \times [5.0 + 1.304 (M_{\rm He} - 64)]$, where
$M_{\rm He} = (13/24) \times (M_\star - 20) \Ms$ is the helium core
(and equivalently the ejecta) mass and $M_\star$ is the stellar mass.

The Pop II stars emit 6000 hydrogen ionizing photons per baryon over
their lifetime, and we do not consider singly- and doubly-ionizing
helium photons.  We note that low-metallicity stars generate up to a
factor of four more ionizing photons than a solar metallicity
population \citep{Schaerer03} and might be underestimating the
radiative feedback.  Nonetheless this study provides an excellent
first insight in the transition to Pop II SF, as the metal enrichment
is the key ingredient here.  For SN feedback, these stars generate
$6.8 \times 10^{48}$ erg s$^{-1}$ $\Ms^{-1}$ after living for 4 Myr,
which is injected into spheres with a radius of 10 pc.  If the
resolution of the grid is less than 10/3 pc, we deposit the energy
into a $3^3$ cube surrounding the star particle.

\subsection{Different physics configurations}

We perform five different simulations with the same initial
conditions.  The fiducial model considers (1) primordial cooling that
is solved with a non-equilbrium chemistry solver with 9 species of
hydrogen, helium, and molecular hydrogen, (2) a Pop III IMF
characteristic mass $M_{\rm char}$ = 100 \Ms, and (3) Pop II and Pop
III star formation.  We vary this model with the following:

\begin{enumerate}
\item Radiative cooling from metal species.  We calculate a cooling
  curve with CLOUDY \citep{CLOUDY} and use the method of
  \citep{2008MNRAS.385.1443S}.  Solar abundances were used in the
  cooling curve.  It is tabulated based on density, temperature,
  electron fraction, and metallicity.
\item Pop III IMF characteristic mass $M_{\rm char}$ = 40 \Ms.
\item Pop II star formation only.
\item Pop II star formation only and no molecular hydrogen cooling.
\end{enumerate}

\section{Results}
\label{sec:results}

\subsection{Halo mass evolution}

\subsection{Metal enrichment}

\subsection{Star formation}

\subsection{Negative feedback from gas loss}

\section{Comparison to Previous Work}

\subsection{Semi-analytical}

\subsection{Numerical}

\section{Discussion}

\li DLAs

\li Bimodality in [Z/H] distribution in globular clusters and dSphs

\li Importance of critical metallicity in satellite halos, not in host
halos

\li [$\alpha$/Fe] estimate

\li Importance of Pop III SF

\section{Summary}

\acknowledgments

J.~H.~W. is supported by NASA through Hubble Fellowship grant
\#120-6370 awarded by the Space Telescope Science Institute, which is
operated by the Association of Universities for Research in Astronomy,
Inc., for NASA, under contract NAS 5-26555.  M.~J.~T. and
M.~L.~N. acknowledge partial support from NASA ATFP grant NNX08AH26G.
Computational resources were provided by NASA/NCCS award SMD-09-1439.
J.~H.~W. thanks Renyue Cen, Amina Helmi, Marco Spaans, and Eline
Tolstoy for enlightening discussions.  The majority of the analysis
and plots were done with \texttt{yt} \citep{yt_full_paper}.

%\clearpage
%\input biblio.tex
\bibliography{ms}

\end{document}
