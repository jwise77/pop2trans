\documentclass[apjl]{emulateapj}
%\documentclass[12pt,preprint]{aastex}

\usepackage{graphicx}
\usepackage{epsfig}
\usepackage{natbib}
\usepackage[section] {placeins}
\bibliographystyle{apj}
\citestyle{apj}

%%%%%%%% Begin custom definitions %%%%%%%%%%%%%

\input macros.tex

%%%%%%%% End custom definitions %%%%%%%%%%

\begin{document}

\shorttitle{THE BIRTH OF A GALAXY}
\shortauthors{WISE ET AL.}

\title{The Birth of a Galaxy: Population III Progenitors and
  Metal-enriched Stellar Populations}

\author{John H. Wise\altaffilmark{1,2}, 
  Michael L. Norman\altaffilmark{3},
  Tom Abel\altaffilmark{4},
  Matthew J. Turk\altaffilmark{3}}

\altaffiltext{1}{Department of Astrophysical Sciences, Princeton
  University, Peyton Hall, Ivy Lane, Princeton, NJ 08544}
\altaffiltext{2}{Hubble Fellow}
\altaffiltext{3}{Center for Astrophysics and Space Sciences,
  University of California at San Diego, La Jolla, CA 92093}
\altaffiltext{4}{Kavli Institute for Particle Astrophysics and
  Cosmology, Stanford University, Menlo Park, CA 94025}
%\altaffiltext{5}{Michigan State University}
\email{jwise@astro.princeton.edu}

\begin{abstract}

  Population III stars first form in dark matter halos with masses
  around $10^6 \Ms$.  By definition, they are metal-free, and their
  protostellar collapse is driven by molecular hydrogen cooling in the
  gas-phase, leading to a massive characteristic mass $\sim 100~\Ms$
  and suppressed fragmentation.  Population II stars with lower
  characteristic masses form when the star-forming gas reaches a
  critical metallicity of $10^{-6} - 10^{-3}~Z_\odot$, depending on
  whether dust cooling is important.  We present adaptive mesh
  refinement radiation hydrodynamics simulations that
  self-consistently follow the transition from Population III to II
  star formation.  We model stellar radiative feedback with adaptive
  ray tracing.  A top-heavy initial mass function for the Population
  III stars is considered, resulting in a plausible distribution of
  pair-instability supernovae and associated metal enrichment.
  \textbf{Add results.}

\end{abstract}

\keywords{cosmology --- methods: numerical --- hydrodynamics ---
  radiative transfer --- star formation}

\section{Motivation}

The first (Pop III) stars are metal-free and have a large
characteristic mass and suppressed fragmentation in its protostellar
collapse \citep{Abel00, ABN02, Bromm02_P3, Yoshida03, OShea07a}.  A
fraction of these stars enrich the surrounding intergalactic medium
(IGM) when they go supernova, which can happen in stars $\lsim40~\Ms$
in Type II supernovae (SNe) or in stars roughly between 140 \Ms~and
260 \Ms~in pair-instability SNe \citep[PISNe;][]{2002ApJ...567..532H}.
The host halo and the neighboring halos are then enriched with this
ejecta.  There exists a critical metallicity that is $\sim 10^{-6}
Z_\odot$ if dust cooling is efficient \citep{Omukai05,
  Schneider06_Frag, clark08} and $\sim 10^{-3.5} Z_\odot$ otherwise
\citep{Bromm01, 2009ApJ...691..441S}, where the gas can cool rapidly,
lowering its Jeans mass.  An intermediate characteristic mass of
$\sim10~\Ms$ can be occur if the gas cooling is suppressed to the
cosmic microwave background (CMB) temperature \citep{Larson98,
  Tumlinson07_IMF, 2009ApJ...691..441S}.  The resulting Population II
star cluster will thus have a lower characteristic stellar mass than
its metal-free progenitors.  In the local universe, stars in the Milky
Way halo and dwarf spheroidal (dSph) galaxies may be connected to
these first stellar clusters.  In the halo of the Milky Way, the most
metal-poor stars have a metallicity floor of [Z/H] = -4
\citep{Beers05}, and in dwarf spheroidal (dSph) galaxies this floor is
a factor of 10 higher \citep{Tolstoy04, Helmi06}.  \textbf{Check for
  more recent references.}

The transition from Pop III to Pop II star formation is solely
dependent on the propagation of metals from the SNe remnants into
future sites of star formation.  Their flows are complex because of
the interactions between the SN blastwave, cosmological accretion and
halo mergers, and nearby stellar feedback.  In minihalos ($M_{\rm h}
\sim 10^6~\Ms$), radiation from the massive Pop III drives a 30
\kms~shock, which is 10 times greater than the escape velocity of the
halo, and leaves behind a warm ($3 \times 10^4$ K) and diffuse (0.5
\cubecm) medium \citep{Kitayama04, Whalen04, Abel07}.  The latter
feature aids in the expansion of the blastwave because it delays the
transition to the Sedov-Taylor and snowplow phases.  In PISNe,
approximately half of the metals stay in the IGM with an metal bubble
size of a few kpc.  The remaining metals fall back into the host halo.
The blastwave may induce star formation in nearby halos through the
compression of the gas \citep{Ferrara98}; however timescales for metal
mixing into the dense gas are many dynamical times \citep{Cen08} for
shock velocities $\lsim 100~\kms$.

Numerical simulations are useful to detangle and study these
complexities and the transition from Pop III to II stars.  In this
Letter, we present the simulations that include both types of star
formation, its radiative feedback, and its SNe feedback.  We utilize
an initial mass function (IMF) for individual Pop III stars, based on
the latest simulation results.  The methods used here incorporate and
link together recent results from metal-enriched and metal-free star
formation, the critical metallicity, and pair-instability supernovae.
Furthermore we spatially distinguish metal enrichment from Pop II and
Pop III stars.  This is the first time that all of these processes
have been interconnected in a simulation that resolves minihalos to
study the transition to Pop II star formation self-consistently.  We
describe our simulation setup and our adopted star formation models in
\S \ref{sec:setup} and present our results in \S \ref{sec:results}.
We discuss the implications of our findings and conclude in the last
section.

\section{Method}
\label{sec:setup}

In this section, we first describe our simulation setup.  We then
detail the star formation models used in these simulations and how we
model stellar radiation and SNe feedback.

\subsection{Simulation setup}

We use the adaptive mesh refinement (AMR) code
\enzo~\citep{BryanNorman1997, OShea2004} that employs our
implementation of adaptive ray tracing for radiation transport
\citep{Wise10}.  To resolve minihalos with at least 100 dark matter
(DM) particles and follow the formation of the first generation of
dwarf galaxies, we use a simulation box of 1 Mpc that has a resolution
of $256^3$.  This gives us a DM mass resolution of 1840 \Ms.  We
refine the grid on baryon overdensities of $3 \times 2^{-0.2l}$, where
$l$ is the AMR level, resulting in a super-Lagrangian behavior.  We
also refine on a DM overdensity of $3 \times 2^l$ and always resolve
the local Jeans length by at least 4 cells.  The latter criterion
avoids any artifical fragmentation during gaseous collapses
\citep{Truelove97}.  We initialize the simulation with \textsl{grafic}
\citep{Bertschinger01} at $z = 130$ and use the cosmological
parameters from the 7-year WMAP data \citep{WMAP7}: $\Omega_M =
0.266$, $\Omega_\Lambda = 0.734$, $\Omega_b = 0.0449$, $h = 0.71$,
$\sigma_8 = 0.81$, and $n = 0.963$ with the variables having their
usual definitions.

We use a non-equilbrium chemistry solver with 9 species of hydrogen,
helium, and molecular hydrogen \citep{Abel97}.  We will follow-up this
study with one that considers radiative cooling from metal-enriched
gas, using rates that are calculated from CLOUDY \citep{CLOUDY} and
the method of \citet{2008MNRAS.385.1443S}.

% We peform two simulations with the same initial conditions.  They
% differ only in that one considers primordial cooling, and the other
% considers additional cooling in metal-enriched gas.  For primordial
% cooling, we use a non-equilbrium chemistry solver with 9 species of
% hydrogen, helium, and molecular hydrogen.  For metal cooling, we use a
% cooling curve that is calculated from CLOUDY \citep{CLOUDY} and using
% the method of \citet{2008MNRAS.385.1443S}.  It is tabulated based on
% density, temperature, electron fraction, and metallicity.

\subsection{Star formation}

We distinguish Pop II and Pop III star formation by the metallicity of
the cold ($T < 10^3$ K) star forming gas in the molecular clouds.  Pop
II stars are formed if [Z/H] $> -4$, and Pop III stars are formed
otherwise.  We do not consider Pop III.2 stars and intermediate mass
stars from CMB-limited cooling.

Simulations have shown that the characteristic mass of Pop III stars
$M_{\rm char} \sim 100~\Ms$.  They form in molecular clouds that
coexist with the dark matter halo center with limited fragmentation
occurring during their collapse; however \citet{2009Sci...325..601T}
has recently shown that Pop III binaries may form in a fraction of
such halos.  

% Furthermore, detailed one-dimensional calculations have shown that
% their IMF should follow a Kroupa-like IMF that has a Salpeter slope
% at the high-mass end and an exponential cut-off below $M_{\rm char}$
% \citep{refs}.

For Pop III stars, we use the same star formation model as
\citet{Abel07} and \citet{Wise08_Gal} where each star particle
represents a single star.  The critical overdensity in which a star
forms is $5 \times 10^5$.  Instead of using a fixed stellar mass, we
randomly sampled from an IMF with a functional form of
%
\begin{equation}
\label{eqn:imf}
f(M)dM = M^{-1.3} \exp\left[-\left(\frac{M_{\rm char}}{M}\right)^{1.6}\right]
\end{equation}
to determine the stellar mass.  Above $M_{\rm char}$, it behaves as a
Salpeter IMF but is exponentially cutoff below that mass
\citep{Chabrier03, Clark09}.  For reproducibility, we initialize both
simulations with the same random seed and record the number of times
the random number generator \citep[Mersenne twister;][]{MTwister} has
been called for use when restarting the simulations.

%%% FOR THE COMPARISON BETWEEN RUNS WITH AND WITHOUT METAL COOLING %%%
%
% Although the Pop III star formation history (SFH) is identical for a
% given $N$ stars, the Pop III SFH for a given halo can be different
% between the two simulations.  This happens because the order in which
% halos form Pop III stars can be altered from enhanced star formation
% and feedback from nearby Pop II star clusters, as we will demonstrate
% in this Letter.

We treat Pop II star formation with the same prescription as
\citet{Wise09}, which is a modified version of the widely used
\citet{Cen92} method but accounts for the fact that the molecular
clouds are resolved.  The critical overdensity is the same as the Pop
III star formation model.  In each star-forming region, a fraction
$c_* = 0.07$ of the cold gas ($T < 10^3$ K) is removed from the grid
and deposited into the star particle.  Each star particle lives for 20
Myr, the maximum lifetime of an OB star.  These stars generate the
majority of the ionizing radiation and SNe feedback in stellar
clusters, thus we ignore lower mass stars.  We discuss our treatment
of stellar feedback next.

\subsection{Stellar feedback}

The mass-dependent luminosities and lifetimes of the Pop III stars are
taken from \citet{Schaerer02}.  The radiation field is evolved with
with adaptive ray tracing \citep{Abel02_RT, Wise10} and is coupled
self-consistently to the hydrodynamics.  We model the \hh~dissociating
radiation with an optically-thin, inverse square profile, centered on
all stars.  These stars die as pair-instability SNe (PISNe) if they
are in the mass range 140--260 \Ms \citep{Heger03}.  We use the
explosion energy from \citet{Heger02}, where we fit the following
function to their models, $E_{\rm PISN} = 10^{51} \times [5.0 + 1.304
(M_{\rm He} - 64)]$, where $M_{\rm He} = (13/24) \times (M_\star - 20)
\Ms$ is the helium core mass and $M_\star$ is the stellar mass.

The Pop II stars emit 6000 hydrogen ionizing photons per baryon over
their lifetime, and we do not consider singly- and doubly-ionizing
helium photons.  We note that low-metallicity stars generate up to a
factor of four more ionizing photons than a solar metallicity
population \citep{Schaerer03} and might be underestimating the
radiative feedback.  Nonetheless this study provides an excellent
first insight in the transition to Pop II star formation, as the metal
enrichment is the key ingredient here.  For SN feedback, these stars
generate $6.8 \times 10^{48}$ erg s$^{-1}$ $\Ms^{-1}$ of energy after
living for 4 Myr, which is injected into spheres with a radius of 10
pc.  If the resolution of the grid is less than 10/3 pc, we deposit
the energy into a $3^3$ cube surrounding the star particle.

\section{Results}
\label{sec:results}

\begin{figure*}
  \plotone{f3}
  \caption{\label{fig:projections} Density-weighted projections of gas
    density (top), temperature (middle), and metallicity (bottom).
    The left column shows the entire simulation volume, where the
    center and right columns focus on the intense and quiet halos,
    respectively.  The metallicity projections are a composite picture
    of metals originating from Pop III (red) and Pop II (blue) stars.}
\end{figure*}

Here we present the gaseous and stellar evolution of two selected
halos in the simulation: one that has an early mass buildup but no
major mergers after $z=12$, and one that experiences a series of major
mergers between $z=10$ and $z=7$.  We name the halos ``quiet'' and
``intense'', respectively.  We start with the birth of the first Pop
III star in its merger history.  We then compare the nature of star
formation in these halos.

We illustrate the state of the simulation at $z=7$ in Figure
\ref{fig:projections} with density weighted projections of gas
density, temperature, and metallicity, showing the entire box and
focusing on the two halos of interest.  The quiet and intense halos
are located in the bottom center and bottom right of the full box,
respectively.

\subsection{Halo evolution}

\begin{figure}
  \epsscale{1.15}
  \plotone{f1}
  \caption{\label{fig:evo} Evolution of the total halo mass (top),
    stellar mass (middle), and gas fraction (bottom) of the quiet
    (dashed) and intense (solid) halos.}
  \epsscale{1}
\end{figure}

\begin{figure}
  \epsscale{1.15}
  \plotone{f2}
  \caption{\label{fig:Zevo} Average stellar metallicities and gas
    metallicities enriched by Pop II and Pop III SNe of the
    intense (top) and quiet (bottom) halos.}
  \epsscale{1}
\end{figure}

%\li Describe the evolution of the baryon fraction and metallicity from
%PISN and Type II SNe metals in the two halos.

Figure \ref{fig:evo} shows the total, metal-enriched stellar, and gas
mass history of the most massive progenitors of both halos.  The quiet
halo undergoes a series of major mergers at $z > 12$, growing by a
factor of 30 to $2.5 \times 10^7 \Ms$ within 150 Myr.  Afterwards it
only grows by a factor of 3 by $z=7$ mainly through smooth accretion
from the filaments and IGM.  It is the most massive halo in the
simulation between redshifts 13 and 10.  At the same time, the intense
halo has a relatively small mass $M = 10^7 \Ms$, but it is contained
in a biased region on a comoving scale of 50 kpc with $\sim30$ halos
with $M > 10^6 \Ms$.  After $z=10$, these halos hierarchically merge
to form a $10^9 \Ms$ halo at $z=7$ with two major mergers at redshifts
10 and 7.9, seen in the rapid increases in total mass in Figure
\ref{fig:evo}.  The merger history of the two halos are not atypical
as dark matter halos can experience both quiescent and vigorous mass
accretion rates in their buildup.

Both halos star forming Pop II stars when $M = 10^7 \Ms$.  This is
consistent with the filtering mass $M_f$ of high-redshift halos when
it accretes mainly from a pre-heated IGM \citep{gnedin98, gnedin00,
  Wise08_Gal}.  The quiet halo forms $10^5 \Ms$ of stars by $z=9$.
This initial starburst photo-evaporates the majority of its molecular
clouds, in addition to heating and ionizing the surrounding IGM out to
a radius of 10--15 kpc at $z=9$.  These respectively reduce the
in-situ and external cold gas supply that could feed future star
formation.

The gas fractions of both halos are depleted from 0.15 to 0.08 by
outflowing shocks driven by ionization fronts in their initial
starbursts.  The quiet halo never recovers to the cosmic fraction
$\Omega_b/\Omega_M$ for the following reason.  The lack of major
mergers of halos above the filtering mass leads to a small gas
fraction, cold gas reservoir, and stellar mass at later times.  This
plays a large role in gas accretion because halos with $M < M_f$ will
be likely photo-evaporated, hosting diffuse warm gas reservoirs
instead of cold dense cores.  During this quienscent period, the halo
mainly accretes warm diffuse gas from the filaments and IGM.  In
contrast, the intense halo grows from major mergers of halos with $M >
M_f$.  The progenitor halos involved in the major mergers are able to
host molecular clouds and have, in general, higher gas fractions.
Between $z=10$ and $z=8$, the gas fraction increases from 0.07 to 0.12
until it rapidly jumps to 0.14 when a gas-rich major merger occurs.
The stellar mass accordingly increases with the ample supply of cold
gas during this period.  The gas fraction then decreases slightly by
0.01 when galactic outflows are generated from the starburst
associated with this major merger.

The evolution of the stellar and gas metallicity of both halos are
illustrated in Figure \ref{fig:Zevo}.  PISNe from Pop III stars enrich
the nearby IGM out to a radius of 10 kpc and provides a metallicity
floor of $[Z_3/H] \sim -3$ in both halos, given $M_{\rm char} = 100
\Ms$.  The metallicity from Pop II SNe initially enrich the ISM of
both halos to an average $[\bar{Z}_2/H]$ between --3.5 and --3.  The
quiet halo is enriched to $[\bar{Z}_2/H] \sim -2.5$ until the
metal-poor inflows balance the metal-rich galactic outflows, which are
directed in the polar directions of the gas disk.  This is apparent in
the metallicity projections in Figure \ref{fig:projections}.  The
average stellar metallicities is equal or slightly higher up to 0.5
dex than $[\bar{Z}_2/H]$.

At $z \sim 8$, the metallicity of the intense halo dramatically
self-enriches itself by a factor of 30 to $[\bar{Z}_2/H] = 1.5$ during
a starburst.  Because this halo is located in a large-scale
overdensity, the ejecta falls back into the halo after reaching
distances up to 20 comoving kpc, in turn keeping the halo gas
metallicity high because the inflows are relatively metal-rich
themselves.  Interestingly the first few Pop II star clusters have
[Z/H] between --1 and --2 and dominate the average stellar metallicity
at $z > 8$.  We further detail the formation of these clusters in
\S\ref{sec:pop}.  After the $z=8$ starburst, the average stellar
metallicity follows the average gas metallicity within 0.1 dex.

\subsection{Population III star formation history}

\li Report on the Pop III stars forming in halo progenitors

The most massive progenitor of the quiet halo interestingly never
hosts a Pop III star.  Instead a nearby halo forms a Pop III star,
which (randomly) produces a PISNe at $z=16$.  The associated blastwave
overruns the most massive progenitor, and the dense core survives this
event and is enriched by this PISN.

\subsection{Population II stellar populations}
\label{sec:pop}

\begin{figure*}
  \plottwo{f4a}{f4b}
  \caption{\label{fig:pops} The scatter plots show the star formation
    history of the intense (left) and quiet (right) halos as a
    function of metallicity at $z=7$.  Each circle represents a star
    cluster, whose area is proportional to its mass.  The open circles
    in the upper right represent sizes of $10^3$, $10^4$, and $10^5$
    \Ms~star clusters.  The upper histogram collects the star formation
    history into 25 equal temporal bins.  The right histogram depicts
    the stellar metallicity distribution.}
\end{figure*}

\li Present results on the star formation history and the physical
reasons for the features seen in the SFH and metallicities.

\li Present the stellar metallicity distribution at the final time

\subsection{Environments for starbursts and quiescent star formation}

\li Present the environment for bursting and quiescent star formation
history

\section{Discussion and Summary}

\acknowledgments

Support for this work was provided by NASA through Hubble Fellowship
grant \#120-6370 awarded by the Space Telescope Science Institute,
which is operated by the Association of Universities for Research in
Astronomy, Inc., for NASA, under contract NAS 5-26555.  Computational
resources were provided by NASA/NCCS award SMD-09-1439.  The majority
of the analysis and plots were done with \texttt{yt}
\citep{yt_full_paper}.

%\clearpage
%\input biblio.tex
\bibliography{ms}

\end{document}
