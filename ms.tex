\documentclass[useAMS,usenatbib]{mn2e}

\usepackage{graphicx}
\usepackage{epsfig}
\usepackage{natbib}
\usepackage[section] {placeins}
\bibliographystyle{mn2e}
\citestyle{mn2e}

%%%%%%%% Begin custom definitions %%%%%%%%%%%%%

\input macros.tex
\voffset=-1.4cm

%%%%%%%% End custom definitions %%%%%%%%%%

\begin{document}

\title[Transition from Pop III to Dwarf Galaxies]{The Transition from
  Massive Metal Free Stars to Dwarf Galaxies}

\author[J. H. Wise et al.]{John H. Wise$^1$\thanks{Hubble Fellow;
    e-mail: jwise@astro.princeton.edu}, Matthew J. Turk$^{2,3}$, Michael
  L. Norman$^3$, Tom Abel$^4$\\
  $^{1}$ Department of Astrophysical Sciences, Princeton University,
  Peyton Hall, Princeton, NJ 08544, USA\\
  $^{2}$ Department of Astronomy, Columbia University, 538 West 120th
  Street, New York, NY 10027, USA\\
  $^{3}$ Center for Astrophysics and Space Sciences,
  University of California at San Diego, La Jolla, CA 92093, USA\\
  $^{4}$ Kavli Institute for Particle Astrophysics and Cosmology,
  Stanford University, Menlo Park, CA 94025, USA}

\pagerange{\pageref{firstpage}--\pageref{lastpage}} \pubyear{2011}

\maketitle
\label{firstpage}

\begin{abstract}

  Population III stars first form in dark matter haloes with masses
  around $10^6 \Ms$.  By definition, they are metal-free, and their
  protostellar collapse is driven by molecular hydrogen cooling in the
  gas-phase, leading to a massive characteristic mass $\sim 100~\Ms$
  and suppressed fragmentation.  Population II stars with lower
  characteristic masses form when the star-forming gas reaches a
  critical metallicity of $10^{-6} - 10^{-3.5}~Z_\odot$, depending on
  whether dust cooling is important.  We present adaptive mesh
  refinement radiation hydrodynamics simulations that follows the
  transition from Population III to II star formation.  We model
  stellar radiative feedback with adaptive ray tracing.  A top-heavy
  initial mass function for the Population III stars is considered,
  resulting in a plausible distribution of pair-instability supernovae
  and associated metal enrichment.  We find that the gas fraction
  recovers from 5 percent to nearly the cosmic fraction in haloes with
  merger histories rich in haloes above $10^7 \Ms$.  A single
  pair-instability supernova is sufficient to enrich the host halo to
  a metallicity floor of $10^{-3} Z_\odot$ and to transition to
  Population II star formation.  This provides a natural explanation
  for the observed floor on damped Lyman alpha (DLA) systems
  metallicities reported in the literature, which is of this order.
  We find that stellar metallicities do not necessarily trace stellar
  ages, as mergers of haloes with established stellar populations can
  create superpositions of $t-Z$ evolutionary tracks.  A bimodal
  metallicity distribution is created after a starburst occurs when
  the halo can cool efficiently through atomic line cooling.

\end{abstract}

\begin{keywords}
  cosmology --- methods: numerical --- hydrodynamics ---
  radiative transfer --- star formation
\end{keywords}

\section{Introduction}

The first generation of stars, so-called Population (Pop) III, in the
universe were metal-free, forming from a primordial mixture of mainly
hydrogen and helium.  Heavy elements and dust grains were not present
in these objects to provide an efficient cooling mechanism in the
collapse of the host molecular cloud, whose collapses primarily rely
on molecular hydrogen line cooling in the gas-phase.  Without any
other cooling channels, the temperatures cannot cool below $\sim$ 200
K, resulting in Jeans masses on the order of 100 \Ms.  Numerical
calculations in the past decade have shown that Pop III stars
have characteristic masses between 30 and 300
\Ms~\citep[e.g.][]{ABN02, Bromm02_P3, OShea07a}, which form in dark
matter (DM) haloes with masses $\sim 10^6 \Ms$
\citep[e.g.][]{MacLow86, Shapiro87, Tegmark97}.

These stars have large luminosities on the order of 10$^6 \lsun$,
surface temperatures of $\sim 10^5$ K, and lifetimes of $\sim3$ Myr
\citep{Bond84, Schaerer02}.  Radiation during their main sequence
creates an \hii~region with $r \sim 1-3~\mathrm{kpc}$, ionizing the
surrounding intergalactic medium as the virial radii of the host
haloes are $r_{\rm vir} \sim 100 \mathrm{pc}$ \citep{Whalen04,
  Kitayama04, Alvarez06, Abel07}.  The over-pressurized \hii~region
drives a $\sim$ 30 \kms~shock, which is 10 times greater than the
escape velocity of a $10^6 \Ms$ halo.  In turn, most of the gas is
expelled from the halo, leaving behind a warm ($3 \times 10^4$ K) and
diffuse (0.5 \cubecm) medium.  In haloes that are encompassed by these
\hii~regions and not completely photo-evaporated \citep{Shapiro04,
  whalen08}, the additional free electrons catalyze both \hh~and HD
molecular line cooling \citep{OShea05, Johnson06, Yoshida07,
  McGreer08}, cooling the gas below $\sim$ 100 K and lowering the
Jeans mass to $\sim10-30$ \Ms.  These lower-mass, metal-free stars
are called Pop III.2, whereas their more massive counterparts
that form in unheated regions are called Pop III.1 stars
\citep{Norman08}.

A fraction of Pop III stars enrich the surrounding IGM when
they go supernova, which can happen in stars $\lsim 40~\Ms$ in Type II
supernovae (SNe) or in stars roughly between 140 \Ms~and 260 \Ms~in
pair-instability SNe \citep[PISNe;][]{2002ApJ...567..532H}. The host
halo and the neighboring haloes are then enriched with this ejecta.
The blastwave expands to a radius of a few kpc \citep{Bromm03_SN,
  Wise08_Gal, Greif10} with approximately half of the ejecta falling
back into the adjacent filaments and haloes.  The blastwave may induce
star formation in nearby haloes through the compression of the gas
\citep{Shapiro87, Ferrara98, Mackey03}, and the timescales for metal
mixing into the dense gas are many dynamical times \citep{Cen08} for
shock velocities $\lsim 100~\kms$.

When star-forming gas reaches some `critical metallicity' $Z_{\rm
  cr}$, radiative cooling from metal fine-structure lines or
\hh~formation on dust grains becomes efficient, allowing cooling to $T
\lsim 100$ K \citep{Omukai05}.  The exact value of $Z_{\rm cr}$ has
been constrained to be between $10^{-6} \zsun$ and $10^{-3.5} \zsun$.
The lower limit applies if dust cooling is important in these
scenarios \citep{Omukai05, Schneider06_Frag, clark08}, and the upper
limit happens if metal line cooling in the gas-phase is the dominant
process \citep{Bromm01, 2009ApJ...691..441S}.  This metal-enriched gas
fragments and most likely forms stars with masses similar to
present-day stellar initial mass functions (IMF), marking the local
transition from Pop III to II star formation.  Intermediate to
these populations and possibly unique to high redshift, there exists a
mode of massive star formation that has a characteristic mass of $\sim
10 \Ms$ \citep{Larson98, Tumlinson07_IMF, 2009ApJ...691..441S}.  This
happens when cooling in metal-enriched gas is limited to the cosmic
microwave background (CMB) temperature $T_{\rm CMB} = 2.73 (1+z)$ K,
which can be significantly higher than the cores of typical molecular
clouds ($T \sim 10$ K).

The transition from Pop III to Pop II star formation is solely
dependent on the propagation of metals from the SNe remnants into
future sites of star formation.  Their flows are complex because of
the interactions between the SN blastwave, cosmological accretion and
halo mergers, and nearby stellar feedback.  Spurred by the notion of a
critical metallicity, this transition has been extensively studied
with (i) volume-averaged semi-analytic models \citep{Scannapieco03,
  Yoshida04, Furlanetto05_Reion}, (ii) models using hierarchical
merger trees \citep{Tumlinson06, Tumlinson10, Salvadori07, Komiya10},
(iii) post-processing of cosmological simulations with blastwave
models \citep{Karlsson08, Trenti09, Trenti10}, and (iv) direct
numerical simulations with stellar feedback \citep{Tornatore07,
  Ricotti08, Maio10_Pop32, Maio10_Enrich}.

Over the past decade, these works have refined the general picture of
inhomogenous metal enrichment and the transition to Pop II star
formation, and here we give a brief overview of its development.
Considering only Pop III star formation, \citeauthor{Yoshida04}
calculated that Pop III stars can raise the mean metallicity to $-4.5
\lsim$ [Z/H]\footnote{We use the conventional notation, [Z/H] $\equiv
  \log(Z/H) - \log(Z_\odot/H_\odot)$.} $\lsim -3.5$ by redshift 15 in
the upper limit where all metal-free stars have $M = 200 \Ms$ and
produce PISNe.  Considering both Pop III and II star formation,
\citeauthor{Scannapieco03} found that the transition between the two
modes is a gradual process where both modes are coeval, confirmed by
most of the later works.  Because the host galaxies have small masses
and are subject to negative radiative feedback through photo-heating
and photo-dissociation, the minimum halo mass gradually increases with
the radiation background intensity, which
\citeauthor{Furlanetto05_Reion} found to delay metal enrichment and
place it closer to the epoch of reionization.  \citeauthor{Trenti09}
noted that underdense regions of the universe are still pristine at
$z=6$ with Pop III stars still forming at these late epochs.  This
group later expanded on these results to find that Pop II SFR becomes
dominant at $z>25$ in the buildup of a MW-type halo.  Furthermore they
stress the importance of a photo-dissociating radiation background
that reduces enrichment by PISNe and increases the importance of
metal-rich galactic outflows, where only $10^{-4} - 10^{-2}$ of PISN
ejecta ends up in EMP stars with $M > 0.9 \Ms$.

The Pop III IMF should play an important role in abundance patterns in
extremely metal poor (EMP) stars in the Milky Way (MW) halo and the
physics of reionization.  Using these data as constraints,
\citeauthor{Tumlinson06} found that Pop III IMFs with log-normal
distributions with mean masses between 8 and 42 \Ms~best fit the data.
Furthermore he concludes that EMP stars with [Z/H] $<$ --3 have
between 1 and 10 metal-free SN progenitors and the Pop III SFR is less
than 1\% of the total SFR at $z=6$.  \citeauthor{Karlsson08} use
observational data of EMP stars with their model to constrain the mass
fraction of Pop III stars that die with a PISN is less than 40 per
cent.  They also conclude that stars enriched primarily by PISNe have
[Ca/H] $\gsim$ --2.6, which could have been missed by some EMP
surveys.

%These first stellar clusters may be connected to stars in the Milky
%Way halo and nearby dwarf spheroidal (dSph) galaxies, both with a
%metallicity floor of [Z/H] = --4 \citep{Beers05, Tafelmeyer10,
%Frebel10_Obs}.
%

Cosmological simulations are useful but computationally expensive to
detangle and study the transition from Pop III to II star formation,
especially the complexities ensuing from hierarchical structure
formation, stellar feedback, and inhomogeneous chemical enrichment.
\citeauthor{Tornatore07} employed smoothed particle hydrodynamics
(SPH) simulations to study this problem and found that Pop III star
formation occurs down to $z=2.5$ in underdense regions.  Their mass
resolution did not allow them to capture star formation in minihaloes
($M_{\rm halo} \sim 10^{5-7} \Ms$) that host Pop III stars, which may
have resulted in an underestimate of metal enrichment.  They found a
`Pop III wave' of star formation, where Pop III star formation is
quenched in an inside-out fashion from biased regions to voids, is
however still valid.  \citeauthor{Ricotti08} used a Lagrangian
moving-mesh simulation with radiative, mechanical, and chemical
feedback that resolved minihaloes to find that (i) 1--10\% of the
cosmic volume is chemically enriched, (ii) the gas reservoir in haloes
with $M < 10^8 \Ms$ are depleted, (iii) there is a large scatter in
the mass-to-light (M/L) ratios of dwarf galaxies, and (iv) their
luminosity function is relatively flat.  Most recently
\citeauthor{Maio10_Pop32} found that the ratios of Pop III to Pop II
SFRs are insensitive to the exact value of $Z_{\rm cr}$.  In contrast
with \citeauthor{Ricotti08}, only $10^{-8}$ of the simulation volume
is chemically enriched by $z=11$.  This difference of several orders
of magnitude may result from the lack of radiative feedback, which can
expel most of the gas from the host halo before the SN explosion.
Supporting this idea, \citet{Whalen08_SN} found that PISN blastwaves
in neutral haloes radiate all of their energy, stalling the shock
within the halo and suppressing outflows.

Here, we present simulations that include both modes of star formation
and their radiative, mechanical, and chemical feedback, extending the
aforementioned previous work on the transition from Pop III to II star
formation and the birth of the first galaxies.  The methods used here
incorporate and link together recent results from metal-enriched and
metal-free star formation, the critical metallicity, and
pair-instability supernovae.  There are some uncertainities in these
input parameters, and we thus run several simulations while varying
these parameters.  This is the first time it has been possible to link
the formation and feedback of the first stars to protogalaxies in a
numerical simulation, resolving the important scales and including the
most important physical effects.  This is the first paper in series
that focuses on the formation and properties of high-redshift dwarf
galaxies.

This paper is structured as follows.  Section \ref{sec:setup}
describes our simulation setup, numerical methods for star formation
and feedback, and different physical models considered in each
simulation.  Next we present our results on the stellar populations
and metal enrichement of the high-redshift dwarf galaxies in Section
\ref{sec:results}.  Then in Section \ref{sec:models}, we show the
changes in galactic properties and metal enrichments in each physical
model.  In Section \ref{sec:compare}, we compare our results with
previous work on the transition to Pop II.  We then discuss the
importance of Pop III star formation on high-redshift galaxy
formation and the implications on the interpretation of damped
Lyman-$\alpha$ absorbers (DLAs) and galactic archaeology in Section
\ref{sec:discuss}.  Finally in Section \ref{sec:summary}, we summarize
our results.

\section{Method}
\label{sec:setup}

In this section, we describe our cosmological radiation hydrodynamics
simulation suite with sub-pc resolution, focusing on first the
formation and feedback of Pop III stars, the transition to Pop II star
formation, and finally the formation of the first galaxies.  We then
detail our star formation and feedback algorithms.  We run a total of
eight different simulations, varying physical parameters.  In the
following description, we quote the parameters used in our fidicual
model and describe their variations later in Sec. \ref{sec:config}.

\subsection{Simulation setup}

We use the adaptive mesh refinement (AMR) code
\textsc{enzo~v2.0}\footnote{\texttt{enzo.googlecode.com, changeset
    b86d8ba026d6}} \citep{OShea2004}.  It uses an $N$-body adaptive
particle-mesh solver \citep{Efstathiou85} to follow the dark matter
(DM) dynamics.  It solves the hydrodynamical equation using the
second-order accurate piecewise parabolic method \citep{Woodward84,
  Bryan95}, while a Riemann solver ensures accurate shock capturing
with minimal viscosity.  We use the recently added HLLC Riemann solver
\citep{Toro94_HLLC} for additional stability in strong shocks and
rarefaction waves.  We use the nine-species (\hi, \hii, \hei, \heii,
\heiii, e$^-$, \hh, \hh$^+$, H$^-$) non-equilibrium chemistry model in
\enzo~\citep{Abel97, Anninos97} and the \hh~cooling rates from
\citet{Glover08_Rates}.

To resolve minihaloes with at least 100 dark matter (DM) particles and
follow the formation of the first generation of dwarf galaxies, we use
a simulation box of $L_{\rm box} = 1$ Mpc that has a resolution of
$256^3$.  This gives us a DM mass resolution of 1840 \Ms.  We refine
the grid on baryon overdensities of $3 \times 2^{-0.2l}$, where $l$ is
the AMR level, resulting in a super-Lagrangian behavior \citep[also
see][]{OShea08}.  We also refine on a DM overdensity of $3 \times 2^l$
and always resolve the local Jeans length by at least 4 cells,
avoiding artificial fragmentation during gaseous collapses
\citep{Truelove97}.  If any of these critera are met in a single cell,
it is flagged for further spatial refinement.

We initialize the simulation with \textsc{grafic}
\citep{Bertschinger01} at $z = 130$ and use the cosmological
parameters from the 7-year WMAP $\Lambda$CDM+SZ+LENS best fit
\citep{WMAP7}: $\Omega_M = 0.266$, $\Omega_\Lambda = 0.734$, $\Omega_b
= 0.0449$, $h = 0.71$, $\sigma_8 = 0.81$, and $n = 0.963$ with the
variables having their usual definitions.  We use a maximum refinement
level of $l = 12$, resulting in a maximal comoving resolution of 1 pc.
We stop the simulation at $z=7$ to avoid any large-scale modes with $r
\sim L_{\rm box}$ entering the non-linear growth regime.  At the final
redshift, this simulation has $1.4 \times 10^8$ computational cells.

\subsection{Star formation}

We consider both Pop II and III star formation in the simulation.  We
distinguish between the two modes by the total metallicity of the
densest cell in the molecular cloud.  Pop II stars are formed if [Z/H]
$> -4$, and Pop III stars are formed otherwise.  We do not consider
Pop III.2 stars and intermediate mass stars from CMB-limited cooling.

Simulations have shown that the characteristic mass of Pop III stars
$M_{\rm char} \sim 100~\Ms$.  They form in molecular clouds that
coexist with the dark matter halo center with limited fragmentation
occurring during their collapse; however \citet{2009Sci...325..601T}
and \citet{Stacy10_Binary} have recently shown that Pop III binaries
may form in a fraction of such haloes.

For Pop III stars, we use the same SF model as \citet{Wise08_Gal}
where each star particle represents a single star, forming at an
overdensity of $5 \times 10^5$.  The overdense cell must have a
converging velocity flow ($\nabla \cdot v < 0$) and be cold ($T <
10^3$ K), similar to the original \citet{Cen92} model.  We do not use
the $t_{\rm cool} < t_{\rm dyn}$ condition either because these
timescales are roughly equivalent during the quasi-static collapse of
a metal-free molecular cloud \citep[e.g.][]{OShea07a}.  Thus this
comparison does not pose a good test whether a volume will collapse
and form a Pop III star.  In its replacement, we require that the cell
have a \hh~fraction $f_{\rm H2} > 5 \times 10^{-3}$ to suppress star
formation in the presence of a strong \hh-dissociating (Lyman-Werner;
hereafter LW) background.  Lastly we do not use the original criterion
of a cell being Jeans unstable because the local Jeans length is
always resolved by at least four cells in our simulation.  Once a cell
satisifies these four criteria, the cell is flagged to form a star
particle, acting as a radiative point source.  To conserve mass within
the simulation, we remove half of the gas in a sphere centered on the
overdense cell that contains twice the stellar mass.  If multiple star
particles form in the same timestep within $r_{\rm merge} = 10$ pc,
these particles are merged, and the resulting position is their center
of mass.

Instead of using a fixed stellar mass as in \citet{Wise08_Gal}, we
randomly sampled from an IMF with a functional form of
%
\begin{equation}
\label{eqn:imf}
f(M)dM = M^{-1.3} \exp\left[-\left(\frac{M_{\rm
        char}}{M}\right)^{1.6}\right] dM
\end{equation}
to determine the stellar mass.  Above $M_{\rm char}$, it behaves as a
Salpeter IMF but is exponentially cutoff below that mass
\citep{Chabrier03, Clark09}.

For reproducibility, we record the number of times the random number
generator \citep[Mersenne twister;][]{MTwister} has been called for use
when restarting the simulations.  Although the Pop III star formation
history (SFH) in the entire simulation is identical for a given $N$
stars, the Pop III SFH for a given halo can be different between the
two simulations.  This happens because the order in which haloes form
Pop III stars can be altered from enhanced star formation and feedback
from nearby Pop II star clusters.

We treat Pop II star formation with the same prescription as
\citet{Wise09}, which is similar to the Pop III prescription but
without the \hh~fraction floor that is removed because the
metal-enriched gas can efficiently cool even in the presence of a
strong UV radiation field \citep[e.g.][]{Safranek10}.  This method is
effectively a modified version of the \citet{Cen92} method but
accounts for the fact that the molecular clouds are resolved.  In the
simulation, we define the molecular cloud as a sphere, centered on the
most dense cell, with a dynamical time $t_{\rm dyn} = 3$ Myr
(corresponding to an average density $\bar{\rho}_{\rm MC} \simeq
1000\mu \cubecm$) and a radius $R_{\rm MC}$.  In this sphere, a
fraction $c_\star = 0.07 f_{\rm cold}$ of the cold gas ($T < 10^3$ K)
is converted into a star particle with mass $m_\star = c_\star
(4\pi/3) \bar{\rho}_{\rm MC} R_{\rm MC}^3$.  We then replace the
sphere with a uniform density $\rho_{\rm MC} = (1 - c_\star) /
(Gt_{\rm dyn}^2)$ and temperature $T = 10^4$ K, which approximates the
initial stages of an \hii~region.  Similar to the Pop III treatment,
we merge any newly created particles within a radius $R_{\rm merge}$.
We set the minimum mass of a star particle to $m_{\rm \star, min} =
1000 \Ms$.  If the initial mass does not exceed $m_{\rm \star, min}$,
the star particle does not provide any feedback and continues to
accrete until it reaches $m_{\rm \star, min}$.  The star particles
live for 20 Myr, the maximum lifetime of an OB star.  These stars
generate the majority of the ionizing radiation and SNe feedback in
stellar clusters, thus we ignore any feedback from lower mass stars.

\subsection{Stellar feedback}

The radiation field is evolved with adaptive ray tracing
\citep{Abel02_RT, Wise10} that is based on the HEALPix framework
\citep{HEALPix} and is coupled self-consistently to the hydrodynamics.
Each star particle is a point source of hydrogen ionizing radiation
with the ionizing luminosity equally split between 48 initial rays
(HEALPix level 2).  We use a mono-chromatic spectrum for the radiation
with the energy $E_{\rm ph}$ equaling the luminosity-weighted photon
energy of the spectrum.  For a cosmological simulation that focuses on
galaxies, this does not significantly affect the overall galactic
dynamics \citep[see Sec. 6.3 in][]{Wise10}.  We do not consider helium
singly- and doubly-ionizing radiation.  As the rays propagate from the
source or into a high resolution AMR grid, they are adaptively split
into 4 child rays, increasing the angular resolution of the solution,
when the solid angle of the ray $\theta = 4\pi/(12 \times 4^{L})$ is
larger than 1/3 of the cell area, where $L$ is the HEALPix level.  We
model the \hh~dissociating radiation with an optically-thin, inverse
square profile, centered on all star particles.  In haloes with $M
\gsim 10^8 \Ms$, the \hh~column density may become large enough to
self-shield itself from LW radiation \citep{Wise08_Gal}.  Thus we may
be underestimating the SFRs in our work.

We use mass-dependent hydrogen ionizing and LW photon luminosites and
lifetimes of the Pop III stars from \citet{Schaerer02}.  We use a
mass-independent photon energy $E_{\rm ph} = 29.6$ eV, appropriate for
the near-constant $10^5$ K surface temperatures of Pop III stars.
They die as Type II SNe if $11 \le M_\star/\Ms \le 40$
\Ms~\citep{Woosley95} and as PISNe if they are in the mass range
140--260 \Ms~\citep{Heger03}, where $M_\star$ is the stellar mass.
For normal Type II SNe between 11--20 \Ms, we use an explosion energy
of \tento{51}~erg and a linear fit to the metal ejecta mass calculated
in \citet{Nomoto06},
%
\begin{equation}
  \label{eqn:typeii}
  M_{\rm Z}/\Ms = 0.1077 + 0.3383 \times (M_\star/\Ms - 11).
\end{equation}
%
We model the SNe of stars with $20 \le M_\star/\Ms \le 40$ as
hypernova with the energies and ejecta masses also taken from
\citeauthor{Nomoto06}, linearly interpolating their results to
$M_\star$.  For PISNe, we use the explosion energy from
\citet{Heger02}, where we fit the following function to their models,
%
\begin{equation}
  \label{eqn:pisn}
  E_{\rm PISN} = 10^{51} \times [5.0 + 1.304 (M_{\rm He}/\Ms - 64)] \; \mathrm{erg},
\end{equation}
%
where $M_{\rm He} = (13/24) \times (M_\star - 20) \Ms$ is the helium
core (and equivalently the metal ejecta) mass and $M_\star$ is the
stellar mass.  The blastwave is modeled by injecting the explosion
energy and ejecta mass into a sphere of 10 pc, smoothed at its surface
to improve numerical stability \citep{Wise08_Gal}.  Because we resolve
the blastwave relatively well with several cells across at its
initialization, the thermal energy is converted into kinetic energy
and agrees with the Sedov-Taylor solution \citep[e.g.][]{Greif07}.

The Pop II stars emit 6000 hydrogen ionizing photons per baryon over
their lifetime and $E_{\rm ph} = 21.6$ eV, appropriate for a [Z/H] =
$-1.3$ population.  We note that lower-metallicity stars generate up
to 60\% more ionizing photons and might be underestimating the
radiative feedback \citep{Schaerer03}.  Nonetheless this study
provides an excellent first insight in the transition to Pop II SF, as
the metal enrichment is the key ingredient here.  For SN feedback,
these stars generate $6.8 \times 10^{48}$ erg s$^{-1}$ $\Ms^{-1}$
after living for 4 Myr, which is injected into spheres with a radius
of 10 pc, using the same technique as the Pop III SNe.  However if the
resolution of the grid is less than 10/3 pc, we distribute the energy
into a $3^3$ cube surrounding the star particle.  In addition to
numerical stability, energy distribution across several cells instead
of a single cell has been shown to match cosmic SFRs better in galaxy
simulations at lower redshifts \citep{Smith10}.

\subsection{Different physics configurations}
\label{sec:config}

\begin{figure}
  \plotone{uvb.eps}
  \caption{\label{fig:uvb} Time evolution of the Lyman-Werner
    background, calculated with the \citet{Wise05} model.}
\end{figure}

We perform eight different simulations with the same initial
conditions.  The fiducial model considers (a) primordial cooling with
9 species of hydrogen, helium, and molecular hydrogen, (b) a Pop III
IMF characteristic mass $M_{\rm char}$ = 100 \Ms, and (c) Pop II and
Pop III star formation.  We vary this model with the following
changes:

\begin{enumerate}
\item Radiative cooling from metal species.  We calculate a cooling
  curve with CLOUDY \citep{CLOUDY} and use the method of
  \citep{2008MNRAS.385.1443S}.  Solar abundances were used in the
  cooling curve.  It is tabulated based on density, temperature,
  electron fraction, and total metallicity.
\item A critical metallicity of [Z/H] = --5 and --6, accounting for
  the case where dust cooling can induce low-mass fragmentation in
  these lower metallicities \citep{Omukai05, Schneider06_Frag}.
\item Pop III IMF characteristic mass $M_{\rm char}$ = 40 \Ms.  Using
  the abundance patterns of EMP stars and reionization as constaints,
  the data are best fit with IMFs favoring stars with masses between
  20--130 \Ms \citep{Umeda03, Tumlinson06}.
\item Lyman-Werner radiation background.  We use the semi-analytical
  model of \citet{Wise05}, updated with the 7-year WMAP parameters and
  optical depth to Thompson scattering, to calculate the LW background
  intensity, plotted in Figure \ref{fig:uvb}.  In this semi-analytical
  model, we use a Pop III stellar mass of 100 \Ms, star formation
  efficiency of 0.005, and escape fraction of 0.2 \citep{Wise09}.  The
  LW radiation from point sources are added to the background.  The
  intensity decreases after $z \sim 14$ because Pop II star formation
  becomes dominant, which produce less LW specific luminosity than Pop
  III stars that have surface temperatures of $T = 10^5$ K.  For
  computational convenience, we fit the background evolution with the
  function
%
  \begin{equation}
    \label{eqn:uvb}
    \log J_{21}(z) = A + Bz - Cz^2 + Dz^3 - Ez^4,
  \end{equation}
%
  where $(A,B,C,D,E) = (-2.356, 0.4562, 0.02680, 5.882 \times 10^{-4},
  -5.056 \times 10^{-6})$ and $J_{21}$ is the specific intensity in
  units of $10^{-21}$ \emis.  This fits the
  model data within 1\% in $6 \le z \le 30$ and is consistent with
  $J_{21}$ values in \citet{Trenti09_SFR}.
\item Pop II star formation only.  This model is useful to study the
  importance of treating Pop III star formation as single stars and
  their pair-instability and Type II SNe.
\item Pop II star formation only and no molecular hydrogen cooling.
  This model provides a good comparison to larger scale simulations
  where it is not computationally feasible to consider \hh~cooling or
  resolve the necessary scales for Pop III star formation.
\end{enumerate}

By running several simulations with different physical scenarios, we
can quantify the importance of each ingredient in high-redshift dwarf
galaxy formation.  This approach is similar in spirit with many other
galaxy formation simulation suites \citep[e.g.][]{Schaye10,
  Maio10_Pop32}.

\section{Results}
\label{sec:results}

\begin{figure*}
  \epsscale{2}
  \plotone{bmosaic.eps}
  \epsscale{1}
  \caption{\label{fig:evo-mosaic} Evolution of the entire simulation
    volume at redshifts 15, 12, 10, 8, and 7 (left to right).
    Pictured here are the density-weighted projections of density
    (top), temperature (middle), and metallicity (bottom).  Note how
    the stellar radiative feedback from low-mass galaxies reionize
    the majority of the volume.  The metallicity projections are a
    composite image of metals originating from Pop II (blue) and III
    (red) stars with magneta indicdating a mixture of the two.}
\end{figure*}

\begin{figure}
  \plotone{Lstar.eps}
  \caption{\label{fig:mass_fn} The top panel shows the halo mass
    function of the simulated haloes (points) where the error bars are
    Poisson noise and the solid line is the analytic halo mass
    function of \citet{Warren06}.  The bottom panel depicts the galaxy
    luminosities versus halo DM mass.  Overplotted are constant
    mass-to-light ratios.  The large scatter at low halo masses are
    similar to ones observed in local dSph galaxies.}
\end{figure}

Here we present the gaseous and stellar evolution of the high-redshift
dwarf galaxies that form in our simulations.  To illustrate its global
properties, we show the density-weighted projections of gas density,
temperature, and metallicity at redshifts 15, 12, 10, 8, and 7 in
Figure \ref{fig:evo-mosaic}.  Most apparent is the photo-heating of
IGM over this redshift range, where the ionized fraction is 0.XX at
$z=7$.  The metallicity projections show the growth of metal-enriched
bubbles, starting first with enrichment from Pop III stars.  Then as
time progresses, the haloes enriched within these bubbles above $Z_{\rm
  cr}$ transition to Pop II star formation, enriching their host halo
and surrounding IGM.  In Fig. \ref{fig:mass_fn}, we show the halo mass
function, agreeing well with the analytic halo mass function of
\citet{Warren06}.  We identify the haloes with a friends-of-friends
(FOF) algorithm \citep{Davis85}.  The excess of low-mass haloes below
$M_{\rm DM} = 2 \times 10^6 \Ms$ is most likely associated with the
biased region in the lower-right region in the projections.
Furthermore, the uncertainities in different analytic halo mass
functions \citep[e.g., see][for a comparison]{Reed07} and halo
identification are comparable than these differences.  Also in this
Figure, we display the galaxy luminosities with respect to halo DM
mass found with the FOF halo finder.  Our fidicual model contains 38
galaxies with a total of 3640 Pop II star clusters and captures the
formation of 333 Pop III stars.  Note that the most massive halo
($M_{\rm DM} = 5.2 \times 10^8 \Ms$) is undergoing a starburst at
$z=7$ and has a high $M/L = 3$.  The lower mass galaxies show a spread
in M/L over two orders of magnitude, similar to the observations of
local dSph galaxies \citep[e.g.][]{Strigari08}.

We split our presentation into two sets of analyses.  In one, we focus
on the evolution of the individual halo and galaxy properties of the
10 most massive haloes at $z=7$ to better understand their formation
process and progenitors.  Second, we analyze the statistics of the
star formation and metal enrichment in all 38 galaxies.  We start by
presenting the change in the global halo properties, followed by their
metal enrichment histories.  Lastly we detail their star formation
histories, metallicity distribution functions, and star formation
rates.

\subsection{Halo mass evolution}

To get a visual sense of the mass accretion and metal enrichment
history of the 10 most massive haloes, we show their density-weighted
projections of gas density, temperature, and composite Pop II+III
metallicity at redshifts 7, 8, and 10 in
Fig. \ref{fig:z7-mosaic}--\ref{fig:z10-mosaic}, respectively.  One
common feature of all of the halos represented in these Figure is that
they are embedded in a preheated IGM with the exception of Halo 7 at
$z=10$.  Prior radiative feedback in either the host or nearby halos
created \hii~regions that encompass these halos.  For \hii~regions
without sustained star formation, these so-called relic \hii~regions
recombine and radiatively cool down to $T = 10^3$ K.  Within the
virial radius, one can see that our simulations capture the
multi-phase ISM.  In the density projections, the effects of
photo-evaporation and cloud crushing are clearly seen, which are
caused by the stellar radiative and mechanical feedback, respectively.
The filaments adjacent to the halos are photo-heated to $T = 1-3
\times 10^4$ K, and then their gas escapes into the IGM no longer
being pressure equilibrium with the gravitational potential of the
filament, i.e. Jeans smoothing.  Furthermore, neighboring halos are
photo-evaporated \citep[see][for the level of halo
photo-evaporation]{Wise08_Reion}, resulting in the suppression of
gas-rich satellite haloes.  A good illustration of photo-evaporation
in progress is the cometary structure in the bottom-center of the
density projection of Halo 4 at $z=7$.

\li Something is wrong with the progenitor finder with Halo 9 at $t =
500$ Myr.  See the extreme jump in mass and decrease in gas fraction?

In Fig. \ref{fig:massevo}, we show the total halo mass $M_{\rm tot}$,
stellar mass $M_\star$, and gas fraction $f_{\rm gas} = M_{\rm gas} /
M_{\rm tot}$ of the most massive halos as a function of time.  We
define the halo as a sphere with a radius $r_{200}$, centered on the
DM center-of-mass, with an overdensity of 200 times the cosmic mean
density.  We identify the progenitors of these halos by searching for
the halo in the previous output with the greatest number of shared
particles (all particles in \enzo~have unique identification numbers).
The mass accretion histories of the halos are not atypical from ones
found in cosmological simulations with periods of smooth accretion
from the IGM and filaments with major mergers occasionally occurs,
appearing as jumps in $M_{\rm halo}(z)$.

Even from these visual representations of the data, it is clear that
almost all of the halos are gas-poor with outflows from Pop III star
formation and PISNe having the greatest effect.


\begin{figure*}
  \plottwo{z7-mosaic0.eps}{z7-mosaic1.eps}
  \caption{\label{fig:z7-mosaic} Density-weighted projections of
    density (left), temperature (center), and metallicity (right) of
    the 10 most massive haloes at $z=7$.  The field of view is 5 proper
    kpc, and the circles have a radius of $r_{200}$.  The composite
    metallicity images are constructed in the same fashion as
    Fig. \ref{fig:evo-mosaic}.}
\end{figure*}

\begin{figure*}
  \plottwo{z8-mosaic0.eps}{z8-mosaic1.eps}
  \caption{\label{fig:z8-mosaic} Same as Fig. \ref{fig:z7-mosaic} but
    at $z=8$, showing the progenitors of the 10 most massive haloes at
    $z=7$.}
\end{figure*}

\begin{figure*}
  \plottwo{z10-mosaic0.eps}{z10-mosaic1.eps}
  \caption{\label{fig:z10-mosaic} Same as Fig. \ref{fig:z8-mosaic} at
    $z=10$.}
\end{figure*}

\begin{figure}
  \plottwo{history0.eps}{history5.eps}
  \caption{\label{fig:massevo} Mass evolution.}
\end{figure}

\subsection{Metal enrichment}

\begin{figure}
  \plotone{Zhistory.eps}
  \caption{\label{fig:Zevo} Metal evolution.}
\end{figure}

\begin{figure}
  \plottwo{z2.eps}{z3.eps}
  \caption{\label{fig:zhalo} Halo metallicities.}
\end{figure}

\begin{figure}
  \plotone{ztot.eps}
  \caption{\label{fig:ztot} Halo metallicities.}
\end{figure}

\subsection{Star formation}

\begin{figure}
  \plotone{mlratios.eps}
  \caption{\label{fig:ML} Mass to light ratios.}
\end{figure}

\begin{figure*}
  \epsscale{2.0}
  \plotone{sfh1.eps}
  \epsscale{1.0}
  \caption{\label{fig:SFR} Star formation history.}
\end{figure*}

\subsection{Unique details about individual haloes}

\section{Differences between Physical Models}
\label{sec:models}

\subsection{Metal cooling}

\subsection{Varying the critical metallicity}

\subsection{Characteristic mass of the Pop III IMF}

\subsection{Lyman-Werner radiation background}

\subsection{Neglecting Pop III star formation}

\subsection{Neglecting Pop III star formation and \hh~formation}

\section{Comparison to Previous Work}
\label{sec:compare}

\subsection{Semi-analytical}

\subsection{Numerical}

\section{Discussion}
\label{sec:discuss}

\li DLAs

\li Bimodality in [Z/H] distribution in globular clusters and dSphs

\li Importance of critical metallicity in satellite haloes, not in host
haloes

\li [$\alpha$/Fe] estimate

\li Importance of Pop III SF

\section{Summary}
\label{sec:summary}

\section*{Acknowledgements}

J.~H.~W. thanks Renyue Cen, Amina Helmi, Marco Spaans, and Eline
Tolstoy for enlightening discussions.  J.~H.~W. is supported by NASA
through Hubble Fellowship grant \#120-6370 awarded by the Space
Telescope Science Institute, which is operated by the Association of
Universities for Research in Astronomy, Inc., for NASA, under contract
NAS 5-26555.  M.~J.~T. and M.~L.~N. acknowledge partial support from
NASA ATFP grant NNX08AH26G.  Computational resources were provided by
NASA/NCCS award SMD-09-1439.  The majority of the analysis and plots
were done with \texttt{yt} \citep{yt_full_paper}.

\bibliography{ms}
\bsp
\label{lastpage}

\end{document}
