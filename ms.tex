\documentclass[apjl]{emulateapj}
%\documentclass[12pt,preprint]{aastex}

\usepackage{graphicx}
\usepackage{epsfig}
\usepackage{natbib}
\usepackage[section] {placeins}
\bibliographystyle{apj}
\citestyle{apj}

%%%%%%%% Begin custom definitions %%%%%%%%%%%%%

\input macros.tex

%%%%%%%% End custom definitions %%%%%%%%%%

\begin{document}

\shorttitle{POPULATION III STAR EPOCH}
\shortauthors{WISE ET AL.}

\title{How Short Lived is the Population III Epoch?}

\author{John H. Wise\altaffilmark{1,2}, 
  Michael L. Norman\altaffilmark{3},
  Tom Abel\altaffilmark{4},
  Matthew J. Turk\altaffilmark{3}}

\altaffiltext{1}{Department of Astrophysical Sciences, Princeton
  University, Peyton Hall, Ivy Lane, Princeton, NJ 08544}
\altaffiltext{2}{Hubble Fellow}
\altaffiltext{3}{Center for Astrophysics and Space Sciences,
  University of California at San Diego, La Jolla, CA 92093}
\altaffiltext{4}{Kavli Institute for Particle Astrophysics and
  Cosmology, Stanford University, Menlo Park, CA 94025}
%\altaffiltext{5}{Michigan State University}
\email{jwise@astro.princeton.edu}

\begin{abstract}

  Population III stars first form in dark matter halos with masses
  around $10^6 \Ms$.  By definition, they are metal-free, and their
  protostellar collapse is driven by molecular hydrogen cooling in the
  gas-phase, leading to a massive characteristic mass $\sim 100~\Ms$
  and suppressed fragmentation.  Population II stars with lower
  characteristic masses form when the star-forming gas reaches a
  critical metallicity of $10^{-6} - 10^{-3}~Z_\odot$, depending on
  whether dust cooling is important.  We present adaptive mesh
  refinement radiation hydrodynamics simulations that
  self-consistently follow the transition from Population III to II
  star formation.  We model stellar radiative feedback with adaptive
  ray tracing.  A top-heavy initial mass function for the Population
  III stars is considered, resulting in a plausible distribution of
  pair-instability supernovae and associated metal enrichment.
  \textbf{Add results.}

\end{abstract}

\keywords{cosmology --- methods: numerical --- hydrodynamics ---
  radiative transfer --- star formation}

\section{Motivation}

The first (Pop III) stars are metal-free and have a large
characteristic mass and suppressed fragmentation in its protostellar
collapse \citep{Abel00, ABN02, Bromm02_P3, Yoshida03, OShea07a}.  A
fraction of these stars enrich the surrounding intergalactic medium
(IGM) when they go supernova, which can happen in stars $\lsim40~\Ms$
in Type II supernovae (SNe) or in stars roughly between 140 \Ms~and
260 \Ms~in pair-instability SNe \citep[PISNe;][]{2002ApJ...567..532H}.
The host halo and the neighboring halos are then enriched with this
ejecta.  There exists a critical metallicity that is $\sim 10^{-6}
Z_\odot$ if dust cooling is efficient \citep{Omukai05,
  Schneider06_Frag, clark08} and $\sim 10^{-3.5} Z_\odot$ otherwise
\citep{Bromm01, 2009ApJ...691..441S}, where the gas can cool rapidly,
lowering its Jeans mass.  An intermediate characteristic mass of
$\sim10~\Ms$ can be occur if the gas cooling is suppressed to the
cosmic microwave background (CMB) temperature \citep{Larson98,
  Tumlinson07_IMF, 2009ApJ...691..441S}.  The resulting Population II
star cluster will thus have a lower characteristic stellar mass than
its metal-free progenitors.  In the local universe, stars in the Milky
Way halo and dwarf spheroidal (dSph) galaxies may be connected to
these first stellar clusters.  In the halo of the Milky Way, the most
metal-poor stars have a metallicity floor of [Z/H] = -4
\citep{Beers05}, and in dwarf spheroidal (dSph) galaxies this floor is
a factor of 10 higher \citep{Tolstoy04, Helmi06}.  \textbf{Check for
  more recent references.}

The transition from Pop III to Pop II star formation is solely
dependent on the propagation of metals from the SNe remnants into
future sites of star formation.  Their flows are complex because of
the interactions between the SN blastwave, cosmological accretion and
halo mergers, and nearby stellar feedback.  In minihalos ($M_{\rm h}
\sim 10^6~\Ms$), radiation from the massive Pop III drives a 30
\kms~shock, which is 10 times greater than the escape velocity of the
halo, and leaves behind a warm ($3 \times 10^4$ K) and diffuse (0.5
\cubecm) medium \citep{Kitayama04, Whalen04, Abel07}.  The latter
feature aids in the expansion of the blastwave because it delays the
transition to the Sedov-Taylor and snowplow phases.  In PISNe,
approximately half of the metals stay in the IGM with an metal bubble
size of a few kpc.  The remaining metals fall back into the host halo.
The blastwave may induce star formation in nearby halos through the
compression of the gas \citep{Ferrara98}; however timescales for metal
mixing into the dense gas are many dynamical times \citep{Cen07} for
shock velocities $\lsim 100~\kms$.

Numerical simulations are useful to detangle and study these
complexities and the transition from Pop III to II stars.  In this
Letter, we present the simulations that include both types of star
formation, its radiative feedback, and its SNe feedback.  We utilize
an initial mass function (IMF) for individual Pop III stars, based on
the latest simulation results.  The methods used here incorporate and
link together recent results from metal-enriched and metal-free star
formation, the critical metallicity, and pair-instability supernovae.
Furthermore we spatially distinguish metal enrichment from Pop II and
Pop III stars.  This is the first time that all of these processes
have been interconnected in a simulation that resolves minihalos to
study the transition to Pop II star formation self-consistently.  We
describe our simulation setup and our adopted star formation models in
\S \ref{sec:setup} and present our results in \S \ref{sec:results}.
We discuss the implications of our findings and conclude in the last
section.

\section{Method}
\label{sec:setup}

In this section, we first describe our simulation setup.  We then
detail the star formation models used in these simulations and how we
model stellar radiation and SNe feedback.

\subsection{Simulation setup}

We use the adaptive mesh refinement (AMR) code
\enzo~\citep{BryanNorman97, OShea2004} that employs our implementation
of adaptive ray tracing for radiation transport \citep{Wise10}.  To
resolve minihalos with at least 100 dark matter (DM) particles and
follow the formation of the first generation of dwarf galaxies, we use
a simulation box of 1 Mpc that has a resolution of $256^3$.  This
gives us a DM mass resolution of 1840 \Ms.  We refine the grid on
baryon overdensities of $3 \times 2^{-0.2l}$, where $l$ is the AMR
level, resulting in a super-Lagrangian behavior.  We also refine on a
DM overdensity of $3 \times 2^l$ and always resolve the local Jeans
length by at least 4 cells.  The latter criterion avoids any artifical
fragmentation during gaseous collapses \citep{Truelove97}.  We
initialize the simulation with \textsl{grafic} \citep{Bertschinger01}
at $z = 130$ and use the cosmological parameters from the 7-year WMAP
data \citep{WMAP7}: $\Omega_M = 0.266$, $\Omega_\Lambda = 0.734$,
$\Omega_b = 0.0449$, $h = 0.71$, $\sigma_8 = 0.81$, and $n = 0.963$
with the variables having their usual definitions.

We use a non-equilbrium chemistry solver with 9 species of hydrogen,
helium, and molecular hydrogen \citep{Abel97}.  We will follow-up this
study with one that considers radiative cooling from metal-enriched
gas, using rates that are calculated from CLOUDY \citep{CLOUDY} and
the method of \citet{2008MNRAS.385.1443S}.

% We peform two simulations with the same initial conditions.  They
% differ only in that one considers primordial cooling, and the other
% considers additional cooling in metal-enriched gas.  For primordial
% cooling, we use a non-equilbrium chemistry solver with 9 species of
% hydrogen, helium, and molecular hydrogen.  For metal cooling, we use a
% cooling curve that is calculated from CLOUDY \citep{CLOUDY} and using
% the method of \citet{2008MNRAS.385.1443S}.  It is tabulated based on
% density, temperature, electron fraction, and metallicity.

\subsection{Star formation}

We distinguish Pop II and Pop III star formation by the metallicity of
the cold ($T < 10^3$ K) star forming gas in the molecular clouds.  Pop
II stars are formed if [Z/H] $> -4$, and Pop III stars are formed
otherwise.  We do not consider Pop III.2 stars and intermediate mass
stars from CMB-limited cooling.

Simulations have shown that the characteristic mass of Pop III stars
$M_{\rm char} \sim 100~\Ms$.  They form in molecular clouds that
coexist with the dark matter halo center with limited fragmentation
occurring during their collapse; however \citet{2009Sci...325..601T}
has recently shown that Pop III binaries may form in a fraction of
such halos.  Furthermore, detailed one-dimensional calculations have
shown that their IMF should follow a Kroupa-like IMF that has a
Salpeter slope at the high-mass end and an exponential cut-off below
$M_{\rm char}$ \citep{refs}.

For Pop III stars, we use the same star formation model as
\citet{Abel07} and \citet{Wise08_Gal} where each star particle
represents a single star.  The critical overdensity in which a star
forms is $5 \times 10^5$.  Instead of using a fixed stellar mass, we
randomly sampled from an IMF with a functional form of
%
\begin{equation}
\label{eqn:imf}
f(M)dM = M^{-1.3} \exp\left[-\left(\frac{M_{\rm char}}{M}\right)^{1.6}\right]
\end{equation}
to determine the stellar mass.  Above $M_{\rm char}$, it behaves as a
Salpeter IMF but is exponentially cutoff below that mass
\citep{Chabrier03, Clark09}.  For reproducibility, we initialize both
simulations with the same random seed and record the number of times
the random number generator \citep[Mersenne twister;][]{ref} has been
called for use when restarting the simulations.  

%%% FOR THE COMPARISON BETWEEN RUNS WITH AND WITHOUT METAL COOLING %%%
%
% Although the Pop III star formation history (SFH) is identical for a
% given $N$ stars, the Pop III SFH for a given halo can be different
% between the two simulations.  This happens because the order in which
% halos form Pop III stars can be altered from enhanced star formation
% and feedback from nearby Pop II star clusters, as we will demonstrate
% in this Letter.

We treat Pop II star formation with the same prescription as
\citet{Wise09}, which is a modified version of the widely used
\citet{Cen92} method but accounts for the fact that the molecular
clouds are resolved.  The critical overdensity is the same as the Pop
III star formation model.  In each star-forming region, a fraction
$c_* = 0.07$ of the cold gas ($T < 10^3$ K) is removed from the grid
and deposited into the star particle.  Each star particle lives for 20
Myr, the maximum lifetime of an OB star.  These stars generate the
majority of the ionizing radiation and SNe feedback in stellar
clusters, thus we ignore lower mass stars.  We discuss our treatment
of stellar feedback next.

\subsection{Stellar feedback}

The mass-dependent luminosities and lifetimes of the Pop III stars are
taken from \citet{Schaerer02}.  The radiation field is evolved with
with adaptive ray tracing \citep{Abel02, Wise10} and is coupled
self-consistently to the hydrodynamics.  These stars die as
pair-instability SNe (PISNe) if they are in the mass range 140--260
\Ms \citep{Heger02}.  We use the explosion energy from
\citeauthor{Heger02}, where we fit the following function to their
models, $E_{\rm PISN} = 10^{51} \times [5.0 + 1.304 (M_{\rm He} -
64)]$, where $M_{\rm He} = (13/24) \times (M_\star - 20) \Ms$ is the
helium core mass and $M_\star$ is the stellar mass.

The Pop II stars emit 6000 hydrogen ionizing photons per baryon over
their lifetime, and we do not consider singly- and doubly-ionizing
helium photons.  We note that low-metallicity stars generate up to a
factor of four more ionizing photons than a solar metallicity
population \citep{Schaerer03} and might be underestimating the
radiative feedback.  Nonetheless this study provides an excellent
first insight in the transition to Pop II star formation, as the metal
enrichment is the key ingredient here.  For SN feedback, these stars
generate $6.8 \times 10^{48}$ erg s$^{-1}$ $\Ms^{-1}$ of energy after
living for 4 Myr, which is injected into spheres with a radius of 10
pc.  If the resolution of the grid is less than 10/3 pc, we deposit
the energy into a $3^3$ cube surrounding the star particle.

\section{Results}
\label{sec:results}

\subsection{Metal enrichment history}

\subsection{Population III star formation history}

\subsection{Population II stellar populations}

\section{Discussion and Summary}

\acknowledgments

J.H.W. is supported by the Hubble Fellowship etc.  The majority of the
analysis and plots were done with \texttt{yt} \citep{yt}.

%\clearpage
%\input biblio.tex
\bibliography{ms}

\end{document}
