% Use only LaTeX2e, calling the article.cls class and 12-point type.

\documentclass[12pt]{article}

\usepackage{graphicx}

\newcommand\aap{{A\&A}}% 
         % Astronomy and Astrophysics 
\newcommand\aaps{{A\&AS}}% 
         % Astronomy and Astrophysics, Supplement 
\newcommand\aj{{AJ}}% 
         % Astronomical Journal 
\newcommand\araa{{ARA\&A}}% 
         % Annual Review of Astron and Astrophys 
\newcommand\apj{{ApJ}}% 
         % Astrophysical Journal 
\newcommand\apjl{{ApJ}}% 
         % Astrophysical Journal, Letters 
\newcommand\apjs{{ApJS}}% 
         % Astrophysical Journal, Supplement 
\newcommand\mnras{{MNRAS}}% 
         % Monthly Notices of the RAS 
\newcommand\nat{{Nature}}% 
         % Nature 
\newcommand\pasp{{PASP}}% 
         % Publications of the ASP 
\newcommand{\cubecm}{\ifmmode{{\rm cm^{-3}}}\else{cm$^{-3}$}\fi}
\newcommand{\lsim}{\lower0.3em\hbox{$\,\buildrel <\over\sim\,$}}
\newcommand{\gsim}{\lower0.3em\hbox{$\,\buildrel >\over\sim\,$}}
\newcommand{\kms}{\ifmmode{{\rm km~s^{-1}}}\else{km s$^{-1}$}\fi}
\newcommand{\hh}{H$_2$}
\newcommand{\Ms}{\ifmmode{M_\odot}\else{$M_\odot$}\fi}
\newcommand\tento[1]{$10^{#1}$}
\newcommand{\tvir}{\ifmmode{T_{\rm{vir}}}\else{$T_{\rm{vir}}$}\fi}

% Users of the {thebibliography} environment or BibTeX should use the
% scicite.sty package, downloadable from *Science* at
% www.sciencemag.org/about/authors/prep/TeX_help/ .
% This package should properly format in-text
% reference calls and reference-list numbers.

%% We will uncomment this later
\usepackage{scicite}
\bibliographystyle{Science}

% Use times if you have the font installed; otherwise, comment out the
% following line.

%\usepackage{times}

% The preamble here sets up a lot of new/revised commands and
% environments.  It's annoying, but please do *not* try to strip these
% out into a separate .sty file (which could lead to the loss of some
% information when we convert the file to other formats).  Instead, keep
% them in the preamble of your main LaTeX source file.


% The following parameters seem to provide a reasonable page setup.

\topmargin 0.0cm
\oddsidemargin 0.2cm
\textwidth 16cm 
\textheight 21cm
\footskip 1.0cm

\begin{document} 

% Double-space the manuscript.

\baselineskip24pt


\section*{Supporting Online Material}

\section*{Method}
\label{sec:setup}

Here, we first describe our simulation setup.  We then detail the SF
models used here and how we model radiative and SNe feedback.

\subsection*{Simulation setup}

We use the adaptive mesh refinement (AMR) code
enzo~v2.0\footnote{\texttt{enzo.googlecode.com, unstable changeset
    b86d8ba026d6}} \cite{BryanNorman1997, OShea2004}, which has been
modified to use a HLLC Riemann solver \cite{Toro94_HLLC} for
additional stability in strong shocks.  To resolve minihalos with at
least 100 dark matter (DM) particles and follow the formation of the
first generation of dwarf galaxies, we use a simulation box of 1 Mpc
that has a resolution of $256^3$.  This gives us a DM mass resolution
of 1840 \Ms.  We refine the grid on baryon overdensities of $3 \times
2^{-0.2l}$, where $l$ is the AMR level, resulting in a
super-Lagrangian behavior.  We also refine on a DM overdensity of $3
\times 2^l$ and always resolve the local Jeans length by at least 4
cells, avoiding artificial fragmentation during gaseous collapses
\cite{Truelove97}.  We initialize the simulation with \textsl{grafic}
\cite{Bertschinger01} at $z = 130$ and use the cosmological parameters
from the 7-year WMAP data \cite{WMAP7}: $\Omega_M = 0.266$,
$\Omega_\Lambda = 0.734$, $\Omega_b = 0.0449$, $h = 0.71$, $\sigma_8 =
0.81$, and $n = 0.963$ with the variables having their usual
definitions.  We stop the simulation at $z=7$.

We use a non-equilibrium chemistry solver with 9 species of hydrogen,
helium, and molecular hydrogen \cite{Abel97}.  We spatially
distinguish metal enrichment from Pop II and Pop III stars.  We will
follow-up this study with one that considers radiative cooling from
metal-enriched gas, using the method of \cite{2008MNRAS.385.1443S}.

% We peform two simulations with the same initial conditions.  They
% differ only in that one considers primordial cooling, and the other
% considers additional cooling in metal-enriched gas.  For primordial
% cooling, we use a non-equilbrium chemistry solver with 9 species of
% hydrogen, helium, and molecular hydrogen.  For metal cooling, we use a
% cooling curve that is calculated from CLOUDY \cite{CLOUDY} and using
% the method of \cite{2008MNRAS.385.1443S}.  It is tabulated based on
% density, temperature, electron fraction, and metallicity.

\subsection*{Star formation}

We distinguish Pop II and Pop III SF by the total metallicity of the
densest cell in the molecular cloud.  Pop II stars are formed if [Z/H]
$> -4$, and Pop III stars are formed otherwise.  We do not consider
Pop III.2 stars and intermediate mass stars from CMB-limited cooling.

Simulations have shown that the characteristic mass of Pop III stars
$M_{\rm char} \sim 100~\Ms$.  They form in molecular clouds that
coexist with the dark matter halo center with limited fragmentation
occurring during their collapse; however \cite{2009Sci...325..601T}
and \cite{Stacy10_Binary} have recently shown that Pop III binaries
may form in a fraction of such halos.

% Furthermore, detailed one-dimensional calculations have shown that
% their IMF should follow a Kroupa-like IMF that has a Salpeter slope
% at the high-mass end and an exponential cut-off below $M_{\rm char}$
% \cite{refs}.

For Pop III stars, we use the same SF model as \cite{Wise08_Gal} where
each star particle represents a single star, forming at an overdensity
of $5 \times 10^5$.  Instead of using a fixed stellar mass, we randomly
sampled from an IMF with a functional form of
%
\begin{equation}
\label{eqn:imf}
f(M)dM = M^{-1.3} \exp\left[-\left(\frac{M_{\rm char}}{M}\right)^{1.6}\right]
\end{equation}
to determine the stellar mass.  Above $M_{\rm char}$, it behaves as a
Salpeter IMF but is exponentially cutoff below that mass
\cite{Chabrier03, Clark09}.  For reproducibility, we record the number
of times the random number generator (Mersenne twister;
\cite{MTwister}) has been called for use when restarting the
simulations.

%%% FOR THE COMPARISON BETWEEN RUNS WITH AND WITHOUT METAL COOLING %%%
%
% Although the Pop III star formation history (SFH) is identical for a
% given $N$ stars, the Pop III SFH for a given halo can be different
% between the two simulations.  This happens because the order in which
% halos form Pop III stars can be altered from enhanced star formation
% and feedback from nearby Pop II star clusters, as we will demonstrate
% in this Report.

We treat Pop II SF with the same prescription as \cite{Wise09}, which
is a modified version of the \cite{Cen92} method but accounts for the
fact that the molecular clouds are resolved.  The critical overdensity
is the same as the Pop III SF model.  In each star-forming region,
seven percent of the cold gas ($T < 10^3$ K) is removed from the grid
and deposited into the star particle that lives for 20 Myr, the
maximum lifetime of an OB star.  These stars generate the majority of
the ionizing radiation and SNe feedback in stellar clusters, thus we
ignore lower mass stars.

\subsection*{Stellar feedback}

We use mass-dependent luminosities and lifetimes of the Pop III stars
from \cite{Schaerer02}.  The radiation field is evolved with adaptive
ray tracing \cite{Abel02_RT, Wise10} and is coupled self-consistently
to the hydrodynamics.  We model the \hh~dissociating radiation with an
optically-thin, inverse square profile, centered on all stars.  These
stars die as pair-instability SNe (PISNe) if they are in the mass
range 140--260 \Ms~\cite{Heger03}.  We use the explosion energy from
\cite{Heger02}, where we fit the following function to their models,
$E_{\rm PISN} = 10^{51} \times [5.0 + 1.304 (M_{\rm He} - 64)]$, where
$M_{\rm He} = (13/24) \times (M_\star - 20) \Ms$ is the helium core
(and equivalently the ejecta) mass and $M_\star$ is the stellar mass.

The Pop II stars emit 6000 hydrogen ionizing photons per baryon over
their lifetime, and we do not consider singly- and doubly-ionizing
helium photons.  We note that low-metallicity stars generate up to a
factor of four more ionizing photons than a solar metallicity
population \cite{Schaerer03} and might be underestimating the
radiative feedback.  Nonetheless this study provides an excellent
first insight in the transition to Pop II SF, as the metal enrichment
is the key ingredient here.  For SN feedback, these stars generate
$6.8 \times 10^{48}$ erg s$^{-1}$ $\Ms^{-1}$ after living for 4 Myr,
which is injected into spheres with a radius of 10 pc.  If the
resolution of the grid is less than 10/3 pc, we deposit the energy
into a $3^3$ cube surrounding the star particle.

\bibliography{ms}

\end{document}
